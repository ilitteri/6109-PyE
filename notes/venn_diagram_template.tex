\documentclass{article}
% Set operations illustrated with Venn diagrams
% Author: Uwe Ziegenhagen
% This is an expanded version of an example provided by T. Tantau

\usepackage{tikz}
\begin{document}

% Definition of circles
\def\firstcircle{(0,0) circle (1.5cm)}
\def\secondcircle{(0:2cm) circle (1.5cm)}

\colorlet{circle edge}{blue!50}
\colorlet{circle area}{blue!20}

\tikzset{filled/.style={fill=circle area, draw=circle edge, thick},
    outline/.style={draw=circle edge, thick}}

\setlength{\parskip}{5mm}
% Set A and B
\begin{tikzpicture}
    \begin{scope}
        \clip \firstcircle;
        \fill[filled] \secondcircle;
    \end{scope}
    \draw[outline] \firstcircle node {$A$};
    \draw[outline] \secondcircle node {$B$};
    \node[anchor=south] at (current bounding box.north) {$A \cap B$};
\end{tikzpicture}

%Set A or B but not (A and B) also known a A xor B
\begin{tikzpicture}
    \draw[filled, even odd rule] \firstcircle node {$A$}
                                 \secondcircle node{$B$};
    \node[anchor=south] at (current bounding box.north) {$\overline{A \cap B}$};
\end{tikzpicture}

% Set A or B
\begin{tikzpicture}
    \draw[filled] \firstcircle node {$A$}
                  \secondcircle node {$B$};
    \node[anchor=south] at (current bounding box.north) {$A \cup B$};
\end{tikzpicture}

% Set A but not B
\begin{tikzpicture}
    \begin{scope}
        \clip \firstcircle;
        \draw[filled, even odd rule] \firstcircle node {$A$}
                                     \secondcircle;
    \end{scope}
    \draw[outline] \firstcircle
                   \secondcircle node {$B$};
    \node[anchor=south] at (current bounding box.north) {$A - B$};
\end{tikzpicture}

% Set B but not A
\begin{tikzpicture}
    \begin{scope}
        \clip \secondcircle;
        \draw[filled, even odd rule] \firstcircle
                                     \secondcircle node {$B$};
    \end{scope}
    \draw[outline] \firstcircle node {$A$}
                   \secondcircle;
    \node[anchor=south] at (current bounding box.north) {$B - A$};
\end{tikzpicture}

\end{document}