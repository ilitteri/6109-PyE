\documentclass[../../main.tex]{subfiles}

\subsubsection{Vector Aleatorio}
\begin{definition}[Vector Aleatorio]
    Sea $(\Omega, \mathcal{A}, P)$ un espacio de probabilidad, se dice que $\mathbb{X} = (X_{1}, \dots, X_{n})$ es un \textbf{vector aleatorio} de dimensión $n$, si para cada $j = 1, \dots, n$, $X_{j}: \Omega \rightarrow \mathbb{R}$ es una $V.A.$
\end{definition}

\begin{theorem}
    Para todo $\mathbb{X} = (x_{1}, \dots, x_{n}) \in \mathbb{R}^{n}$ se tendrá que
    \begin{equation*}
        X^{-1} = ((-\infty, x_{1}) \cdot \dots \cdot (-\infty, x_{n})) \in \mathcal{A}
    \end{equation*}
\end{theorem}

\subsubsection{Función de Distribución de un Vector Aleatorio}
\subfile{subsubsections/funcion_de_distribucion_vector_aleatorio.tex}

\subsubsection{Función de Probabilidad de un Vector Aleatorio Discreto}
\subfile{subsubsections/funcion_de_probabilidad_vector_aleatorio.tex}

\subsubsection{Función de Probabilidad Marginal Discretas}
\subfile{subsubsections/funcion_de_probabilidad_marginal_discreta.tex}

En general, sea $A$ cualquier conjunto compuesto de pares de valores $(x, y)$ entonces
\begin{equation*}
    P((X, Y) \in A) = \sum\sum_{(x, y) \in A} p_{X, Y}(x, y)
\end{equation*}

\subsubsection{Función de Densidad de un Vector Aleatorio Continuo}
\subfile{subsubsections/funcion_de_densidad_vector_aleatorio.tex}

\subsubsection{Función de Probabilidad Marginal Continuas}
\subfile{subsubsections/funcion_de_probabilidad_marginal_continua.tex}

\subsubsection{Independencia en Variables Aleatorias}
\subfile{subsubsections/independencia_variables_aleatorias.tex}