\documentclass[../../main.tex]{subfiles}

\begin{definition}[\#CP]
    Cantidad de casos posibles de un experimento.
\end{definition}

Si tiramos un dado vemos que la cantidad de resultados posibles es $6$; si tiramos el dado $2$ veces, la cantidad de resultados posibles es $36$; si tiramos el dado $3$ veces, tengo $6$ opciones para cada tiro:

\begin{equation*}
    \underline{6}\text{ }\underline{6}\text{ }\underline{6}
\end{equation*}

\begin{definition}[Regla del Producto]
    Dados dos conjuntos $A$ y $B$ con $\eta_{A}$ y $\eta_{B}$ elementos cada uno respectivamente, la cantidad de todos los pares ordenados que pueden formarse con un elemento de $A$ y uno de $B$ se calcula como $\eta_{A} \cdot \eta_{B}$
\end{definition}

\begin{defexamples} Regla del producto
    \begin{enumerate}
        \item Tiro un dado 2 veces y por regla del producto:
            \begin{equation*}
                \#CP = \underline{6}\cdot\underline{6} = 36
            \end{equation*}
        \item  Tiro un dado 3 veces y por regla del producto:
            \begin{equation*}
                \#CP = \underline{6}\cdot\underline{6}\cdot\underline{6} = 6^{3}
            \end{equation*}
        \item Tiro un dado 4 veces y por regla del producto:
            \begin{equation*}
                \#CP = \underline{6}\cdot\underline{6}\cdot\underline{6}\cdot\underline{6} = 6^{4}
            \end{equation*}
        \item En 4 tiros, ¿de cuántas formas posibles puede no aparecer ningún 6?\\
        Tengo 5 opciones en donde no sale un 6, y como son 4 tiros, entonces tengo 5 opciones en donde no sale un 6, 4 veces. Entonces:
            \begin{equation*}
                \#CP = \underline{5}\cdot\underline{5}\cdot\underline{5}\cdot\underline{5} = 5^{4}
            \end{equation*}
        \item ¿Cuántas patentes de 3 letras y 3 números distintas puedo formar?\\
        Tengo 3 letras (una raya por letra) y 3 números (una raya por número). Tengo 26 letras posibles para la primer posición, para la segunda y tercera igual (porque se pueden repetir). Con los números, tengo 10 opciones para el primero, el segundo y el tercero (10 dígitos de 0 a 9). Por regla del producto:
            \begin{equation*}
                \#CP = \underline{26}\cdot\underline{26}\cdot\underline{26}\cdot\underline{10}\cdot\underline{10}\cdot\underline{10} = 26^{3}\cdot10^{3}
            \end{equation*}
    \end{enumerate}
\end{defexamples}

Supongamos que tengo 5 libros y quiero ordenarlos en una biblioteca, que sólo tiene 5 lugares. Quiero contar todas las formas posibles de ordenarlos. Entonces supongo un caso análogo a los anteriores.\\
Para el primer lugar tengo 5 posibles opciones para colocar uno de mis libros, para el segundo 4 opciones, para el tecero 3 opciones, para el cuarto 2 opciones y para el último lugar ya solo me queda una posible opción. Entonces:

\begin{equation*}
    \#CP = \underline{5}\cdot\underline{4}\cdot\underline{3}\cdot\underline{2}\cdot\underline{1} = 120 = 5!
\end{equation*}

Lo que hicimos fué pensar en la cantidad de libros que teníamos y ver la cantidad de \textit{ordenamientos} posibles.

\begin{definition}[Permutaciones]
    La cantidad de formas distintas en las que puedo ordenar n elementos es n!
\end{definition}

\begin{defexamples} Permutaciones
    \begin{enumerate}
        \item ¿De cuántas formas distintas puedo fotografiar a 7 personas en hilera?
            \begin{equation*}
                \#CP = \underline{7}\cdot\underline{6}\cdot\underline{5}\cdot\underline{4}\cdot\underline{3}\cdot\underline{2}\cdot\underline{1} = 7!
            \end{equation*}
        \item ¿Que pasaría si en una biblioteca tengo espacio para 3 libros y yo tengo 5?\\
        5! son todas las permutaciones de los elementos que tenía originalmente, y hay que dividir por los que conté de más cuando ubiqué a todos. En este caso calculo las permutaciones y luego les divido lo que conté de más, entonces:
            \begin{equation*}
                \#CP = \underline{5}\cdot\underline{4}\cdot\underline{3} = 60 = \frac{5\cdot4\cdot3\cdot2\cdot1}{2\cdot1} = \frac{5!}{2!}
            \end{equation*}
    \end{enumerate}
\end{defexamples}

\begin{definition}[Variaciones]
    Es la cantidad de subconjuntos ordenados de r elementos que se pueden formar a partir de un conjunto de n elementos. Las variaciones son, de un conjunto total de n elementos son todos los subconjuntos diferentes que puedo extraer si me importa el orden en el que los estoy extrayendo.
    \begin{equation*}
        P_{n,r} = \frac{n!}{(n-r)!}
    \end{equation*}
    \begin{equation*}
        \text{En la calculadora: } \boxed{nPr}
    \end{equation*}
\end{definition}

\begin{defexample}
    En un cine con 70 butacas, ¿de cuántas formas distintas pueden sentarse 45 personas?
    \begin{equation*}
        \#CP = \frac{70!}{(70-45)!} = \frac{70!}{25!}
    \end{equation*}
\end{defexample}

\begin{definition}[Combinaciones]
    Es la cantidad de subconjuntos \underline{NO} ordenados de r elementos que pueden formarse a partir de los conjuntos de n elementos.
    \begin{equation*}
        C_{n,r} = \frac{n!}{(n-r)!\cdot r!} = \binom{n}{r}
    \end{equation*}
    \begin{equation*}
        \text{En la calculadora: } \boxed{nCr}
    \end{equation*}
    En donde $\binom{n}{r}$ se denomina número combinatorio de n elementos tomados de a r. Lo que me dice este número es de cuantas formas puedo sacar r elementos de un total de n cuando no me importa el orden.
\end{definition}

\begin{defexamples} Combinaciones
    \begin{enumerate}
        \item ¿De cuántas formas diferentes puedo elegir 11 personas de un grupo de 30 para formar un equipo de fútbol?\\
        Tengo un conjunto de 30 personas, y voy a elegir un subconjunto de 11 personas entonces n = 30, r = 11 y como el orden no me importa, utilizo el número combinatorio para contar estas personas:
            \begin{equation*}   
                \#CP = \binom{30}{11}
            \end{equation*}
        \item Control de calidad. ¿Cuántas muestras de 10 piezas diferentes puedo elegir de un lote de 100?\\
        Tengo un conjunto, y un subconjunto, y el orden no me interesa, entonces:
            \begin{equation*}
                \#CP = \binom{100}{10}
            \end{equation*}
    \end{enumerate}
\end{defexamples}

\begin{observation}
    \begin{equation*}
        \binom{n}{r} = \binom{n}{n-r}
    \end{equation*}
\end{observation}

\begin{proof}
    \begin{equation*}
        \binom{n}{r} = \binom{n}{n-r}
    \end{equation*}
    \begin{equation*}
        \frac{n!}{r!(n-r)!} = \frac{n!}{(n-r)!r!}
    \end{equation*}
\end{proof}

\begin{defexample} Mazo de 40 cartas esáñolas.
    \begin{enumerate}[label=\alph*)]
        \item En un mazo de cartas españolas, tengo 10 cartas de espada, 10 cartas de copa, 10 cartas de basto y 10 cartas de oro. ¿Cuántas manos diferentes puede tener una persona jugando al truco?\\
            \begin{equation*}
                n = 40, r = 3
            \end{equation*}
            \begin{equation*}
                \#CP = \binom{40}{3}
            \end{equation*}
        \item ¿Cuántas formas distintas hay de recibir una mano de oro?\\
        Todos los casos posibles se dan de elegir 3 cartas de un total de 10 ya que estoy elegiendo entre las cartas que son de oro.
            \begin{equation*}
                \#CP = \binom{10}{3}
            \end{equation*}
        \item ¿Cuántas manos puedo recibir siendo todas las cartas del mismo palo?\\
        Separo en casos: tengo 4 posibles resultados dada la cantidad de palos disponibles, y por cada palo tengo la cantidad de posibilidades del resultado anterior, entonces ahora tengo que sumar todas las formas de sacar 3 de cada palo.
            \begin{equation*}
                \#CP_{oro} = \binom{10}{3}, \#CP_{basto} = \binom{10}{3}, \#CP_{espada} = \binom{10}{3}, \#CP_{copa} = \binom{10}{3}
            \end{equation*}
            \begin{equation*}
                \#CP = \#CP_{oro} + \#CP_{basto} + \#CP_{espada} + \#CP_{copa}
            \end{equation*}
            \begin{equation*}
                \#CP = 4 \cdot \#CP_{palo}
            \end{equation*}
            \begin{equation*}
                \#CP = 4 \cdot \binom{10}{3}
            \end{equation*}
    \end{enumerate}
\end{defexample}

\begin{definition}[Anagramas]
    Dada una palabra, un anagrama es una forma de escribir otra palabra, cambiando las letras de la palabra original de lugar. 
\end{definition}

\begin{defexample}
    Supongamos la palabra "ANANA", ¿cuántos anagramas puedo formar?\\
    ¿Cómo contamos todos los ordenamientos posibles? Usando lo que aprendimos en número combinatorio: tengo 5 posiciones para ubicar 5 letras, las letras que tengo que ubicar son las letras de la palabra "ANANA":
    \begin{enumerate}
        \item Acomodo las letras "A", tengo 3 lugares para ubicarla (porque tengo 3). No me importa el orden.
            \begin{equation}
                \#CP_{A} = \binom{5}{3}
            \end{equation}
        
        \item  Por todas esas posiciones que tengo para ubicar la letra "A", tengo que ver todas las posibles opciones que me quedan para la letra "N". Ahora tengo 2 lugares para ubicarla (porque tengo 2).
            \begin{equation}
                \#CP_{N} = \binom{2}{2} = 1
            \end{equation}
    \end{enumerate}
    Una vez ubicadas las letras "A" tengo sólo una forma posible de ubicar las "N". Si ubico primero las "N", de los 5 lugares que tengo voy a poder usar 2, y de los 3 lugares que quedan voy a elegir 3 para ubicar la "A".
    \begin{equation*}
        \#CP = \binom{5}{3} \cdot \binom{2}{2} = \binom{5}{2} \cdot \binom{3}{3} 
    \end{equation*}
\end{defexample}

\begin{note} Otra forma de pensarlo permutando es hacer el total de permutaciones y dividir esa cantidad por la multiplicación de los casos que conté de más
    \begin{equation*}
        \#CP = \frac{5!}{3!\cdot2!} = \binom{5}{3} = \binom{5}{2}
    \end{equation*}
\end{note}

\begin{defexample} ¿Cuántos anagramas puedo formar de la palabra "MANZANA"?
    \begin{itemize}
        \item Con el método de permutaciones:
            \begin{equation*}
                \#CP = \frac{7!}{1!\cdot3!\cdot2!\cdot1!}
            \end{equation*}
        \item Con números combinatorios (leyendo las letras en orden izq-der):
            \begin{equation*}
                \#CP = \binom{7}{1}\cdot\binom{6}{3}\cdot\binom{3}{2}\cdot\binom{1}{1}
            \end{equation*}
    \end{itemize}
\end{defexample}

\begin{definition}[Permutaciones con elementos repetidos]
    Si tenemos n elementos en los cuales hay $n_{1}$ de la clase 1, $n_{2}$ de la clase 2, ... $n_{k}$ de la k-ésima, el número de permutaciones de $n = n_{1} + n_{2} + \dots + n_{k}$ objetos está dada por
    \begin{equation*}
        \frac{n!}{n_{1}!\cdot n_{2}!\cdot\dots n_{k}!}
    \end{equation*}
\end{definition}

\begin{definition}[Método de bolas y urnas (capitulo1-video3 40:00)]
    Se usa cuando tengo una cantidad de elementos \textbf{indistinguibles}. Si estoy ordenando r elementos indistinguibles en k urnas tengo que dibujar r crucecitas ("$\times$"), y k-1 palitos ("$|$")
\end{definition}

\begin{definition}[Métodos de Bose-Einstein]
    \begin{equation*}
        \#CP = \binom{r + k-1}{r} = \binom{r + k - 1}{k - 1}
    \end{equation*}
\end{definition}

\begin{defexample} Tengo un mapa, donde cada cuadrado es una manzana, y quiero ir del punto C al punto S. ¿De cuántas formas posibles puedo ir de C a S si solo puedo moverme hacia la izquierda o abajo.\\
Esto es análogo al problema del anagrama, por lo tanto
    \begin{equation*}
        \#CP = \binom{10}{3} = \binom{10}{7}
    \end{equation*}
\end{defexample}