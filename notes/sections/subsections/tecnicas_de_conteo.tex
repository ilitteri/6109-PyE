\documentclass[../../main.tex]{subfiles}

\subsubsection{Numero de Casos Probales}
\subfile{subsubsections/casos_probables.tex}

Si tiramos un dado vemos que la cantidad de resultados posibles es $6$; si tiramos el dado $2$ veces, la cantidad de resultados posibles es $36$; si tiramos el dado $3$ veces, tengo $6$ opciones para cada tiro:

\begin{equation*}
    \underline{6}\text{ }\underline{6}\text{ }\underline{6}
\end{equation*}

\subsubsection{Regla del Producto}
\subfile{subsubsections/regla_del_producto.tex}

Supongamos que tengo 5 libros y quiero ordenarlos en una biblioteca, que sólo tiene 5 lugares. Quiero contar todas las formas posibles de ordenarlos. Entonces supongo un caso análogo a los anteriores.\\
Para el primer lugar tengo 5 posibles opciones para colocar uno de mis libros, para el segundo 4 opciones, para el tecero 3 opciones, para el cuarto 2 opciones y para el último lugar ya solo me queda una posible opción. Entonces:

\begin{equation*}
    \#CP = \underline{5}\cdot\underline{4}\cdot\underline{3}\cdot\underline{2}\cdot\underline{1} = 120 = 5!
\end{equation*}

Lo que hicimos fué pensar en la cantidad de libros que teníamos y ver la cantidad de \textit{ordenamientos} posibles.

\subsubsection{Permutaciones}
\subfile{subsubsections/permutaciones.tex}

\subsubsection{Variaciones}
\subfile{subsubsections/variaciones.tex}


\subsubsection{Combinaciones}
\subfile{subsubsections/combinaciones.tex}

\subsubsection{Anagramas}
\subfile{subsubsections/anagramas.tex}

\subsubsection{Permutaciones con Elementos Repetidos}
\subfile{subsubsections/permutaciones_elementos_repetidos.tex}

\subsubsection{Metodo de las Bolas y Urnas}
\subfile{subsubsections/metodo_bolas_urnas.tex}

\subsubsection{Metodo Bose-Einstein}
\subfile{subsubsections/metodo_bose_einstein.tex}