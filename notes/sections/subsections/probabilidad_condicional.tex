\documentclass[../../main.tex]{subfiles}

Se trata de analizar cómo afecta la información de que "un evento B ha ocurrido" a la probabilidad asignada de A.

\begin{definition}
    Sea $(\Omega, \mathcal{A}, P)$ un espacio de probabilidad, $A, B \in \mathcal{A}$ con $P(B)> 0$, la probabilidad condicional de $A$ dado que $B$ ha ocurrido está definida por:
        \begin{equation*}
            P(A|B) = \frac{P(A \cap B)}{B}
        \end{equation*}
\end{definition}

\begin{properties}
    La $P(A|B)$ para un $B$ fijo satisface todos los axiomas de probabilidad:
    \begin{enumerate}
        \item $0 \leq P(A|B) \leq 1$, $\forall A \in \mathcal{A}$
            \begin{proof}
                $$P(A|B) = \frac{P(A \cap B)}{B} \leq 1$$
                $$\text{Si } A \cap B = \O \rightarrow P(A \cap B) = 0$$
                $$\begin{aligned}
                    \text{Si } A \cap B \neq \O &\rightarrow A \cap B \subseteq B\\
                    &\rightarrow P(A \cap B) \leq P(B)
                \end{aligned}$$
            \end{proof}
        \item $P(\Omega|B) = 1$
            \begin{proof}
                $$P(\Omega|B) = \frac{\Omega \cap B}{P(B)} = \frac{P(B)}{P(B)} \because B \subseteq \Omega$$
            \end{proof}
        \item Si $A \cap C = \O \rightarrow P(A \cup C|B) = P(A|B) + P(C|B)$
        \item Si $P(B) > 0$
            \begin{enumerate}
                \item $P(A \cap B) = P(A|B) \cdot P(B) = P(B|A) \cdot P(A)$.
                \item $P(A \cap B \cap C) = P(A|B \cap C) \cdot P(B \cap B) = P(A|B \cap C) \cdot P(B|C) \cdot P(C)$
            \end{enumerate}
    \end{enumerate}

    \begin{examples*}
        \begin{enumerate}
            \item Si un programador usa C ¿cuál es la probabilidad de que use Java?\\
                Tenemos que $P(C) > 0$, y que lo que sabes es $P(C)$, y buscamos $P(J)$ entonces
                \begin{equation*}
                    P(J|C) = \frac{P(J \cap C)}{P(C)} = \frac{0 \cdot 1}{0 \cdot 4} = \frac{1}{4}
                \end{equation*}
            \item Si un programador no usa Java, ¿cuál es la probabilidad de que use C?\\
                Nuevamente identificamos que lo que me está pidiendo es una probabilidad condicional. Con la información que tenemos
                \begin{equation*}
                    P(C|\overline{J}) = \frac{P(C \cap \overline{J})}{P(\overline{J})} = \frac{0 \cdot 3}{0 \cdot 5} = \frac{3}{5}
                \end{equation*}
            \item Una persona arroja 2 dados equilibrados. Calcular la probabilidad de que la suma sea 7 dado que:
                \begin{enumerate}
                    \item La suma es impar.
                    \item La suma es mayor que 6.
                    \item El número del $2^{do}$ dado es impar.
                    \item El número de alguno de los dados es impar.
                    \item Los dos números son iguales.
                \end{enumerate}
                Defino mi EA: "Arrojo dos dados y observo". Ahora defino un evento $D_{i}$: "Valor observado en el dado $i$", $i= 1, 2$.
                \begin{enumerate}[label=(\alph*)]
                    \item Debo calcular la probabilidad de que la suma sea 7 dado que la suma es impar, dicho de otra manera, yo ya se que la suma es impar, y a partir de ello quiero calcular la probabilidad de que la suma sea 7.
                    \begin{table}[H]
                        \begin{center}
                            \begin{tabular}{c|c|c|c|c|c|c}
                                $D_{2}/D_{1}$ & $1$ & $2$ & $3$ & $4$ & $5$ & $6$\\
                                \hline
                                $1$ & \cellcolor{white} & \cellcolor{red} & \cellcolor{white} & \cellcolor{red} & \cellcolor{white} & \cellcolor{orange}\\
                                \hline
                                $2$ & \cellcolor{red} & \cellcolor{white} & \cellcolor{red} & \cellcolor{white} & \cellcolor{orange} & \cellcolor{white}\\
                                \hline
                                $3$ & \cellcolor{white} & \cellcolor{red} & \cellcolor{white} & \cellcolor{orange} & \cellcolor{white} & \cellcolor{red}\\
                                \hline
                                $4$ & \cellcolor{red} & \cellcolor{white} & \cellcolor{orange} & \cellcolor{white} & \cellcolor{red} & \cellcolor{white}\\
                                \hline
                                $5$ & \cellcolor{white} & \cellcolor{orange} & \cellcolor{white} & \cellcolor{red} & \cellcolor{white} & \cellcolor{red}\\
                                \hline
                                $6$ & \cellcolor{orange} & \cellcolor{white} & \cellcolor{red} & \cellcolor{white} & \cellcolor{red} & \cellcolor{white}\\
                            \end{tabular}
                        \end{center}
                    \end{table}
                    \begin{equation*}
                        P(\underbrace{D_{1}+D_{2} = 7}_{\colorbox{yellow}{S}}|\underbrace{\text{"La suma es impar"}}_{\colorbox{red}{A}})
                    \end{equation*}
                    \begin{equation*}
                        \begin{aligned}
                            {P(S|A)} &= \frac{P(S\cap A)}{P(A)}\\
                                     &= \frac{\frac{6}{36}}{\frac{18}{36}}\\
                                     &= \frac{6}{18}
                        \end{aligned}
                    \end{equation*}
                    Este resultado se puede interpretar como, de los 18 cuadraditos verdes, 6 son naranjas. En otras palabras, cuando tengo un espacio equiprobable y condiciono, es decir, quito posbiles resultados de mi experimento, los resultados restantes siguen siendo equiprobables, lo que me permite seguir calculando la probabilidad con Laplace. Esto hace que en espacios equiprobables hacer las cuentas sea mucho mas fácil.
                    \item \colorbox{cyan}{B}: "La suma es mayor que 6"
                    \begin{table}[H]
                        \begin{center}
                            \begin{tabular}{c|c|c|c|c|c|c}
                                $D_{2}/D_{1}$ & $1$ & $2$ & $3$ & $4$ & $5$ & $6$\\
                                \hline
                                $1$ & \cellcolor{white} & \cellcolor{red} & \cellcolor{white} & \cellcolor{red} & \cellcolor{white} & \cellcolor{violet}\\
                                \hline
                                $2$ & \cellcolor{red} & \cellcolor{white} & \cellcolor{red} & \cellcolor{white} & \cellcolor{violet} & \cellcolor{cyan}\\
                                \hline
                                $3$ & \cellcolor{white} & \cellcolor{red} & \cellcolor{white} & \cellcolor{violet} & \cellcolor{cyan} & \cellcolor{cyan}\\
                                \hline
                                $4$ & \cellcolor{red} & \cellcolor{white} & \cellcolor{violet} & \cellcolor{cyan} & \cellcolor{cyan} & \cellcolor{cyan}\\
                                \hline
                                $5$ & \cellcolor{white} & \cellcolor{violet} & \cellcolor{cyan} & \cellcolor{cyan} & \cellcolor{cyan} & \cellcolor{cyan}\\
                                \hline
                                $6$ & \cellcolor{violet} & \cellcolor{cyan} & \cellcolor{cyan} & \cellcolor{cyan} & \cellcolor{cyan} & \cellcolor{cyan}\\
                            \end{tabular}
                        \end{center}
                    \end{table}
                    De los 21 casos en donde la suma es mayor a 6, sólo en 6 la suma es 7
                    \begin{equation*}
                        \begin{aligned}
                            {P(S|B)} &= \frac{P(S\cap B)}{P(B)}\\
                                     &= \frac{6}{21}\\
                        \end{aligned}
                    \end{equation*}
                    \item \colorbox{yellow}{C}: "El $2^{do}$ dado es impar"
                    \begin{table}[H]
                        \begin{center}
                            \begin{tabular}{c|c|c|c|c|c|c}
                                $D_{2}/D_{1}$ & $1$ & $2$ & $3$ & $4$ & $5$ & $6$\\
                                \hline
                                $1$ & \cellcolor{yellow} & \cellcolor{orange} & \cellcolor{yellow} & \cellcolor{orange} & \cellcolor{yellow} & \cellcolor{orange}\\
                                \hline
                                $2$ & \cellcolor{red} & \cellcolor{white} & \cellcolor{red} & \cellcolor{white} & \cellcolor{red} & \cellcolor{white}\\
                                \hline
                                $3$ & \cellcolor{yellow} & \cellcolor{orange} & \cellcolor{yellow} & \cellcolor{orange} & \cellcolor{yellow} & \cellcolor{orange}\\
                                \hline
                                $4$ & \cellcolor{red} & \cellcolor{white} & \cellcolor{red} & \cellcolor{white} & \cellcolor{red} & \cellcolor{white}\\
                                \hline
                                $5$ & \cellcolor{yellow} & \cellcolor{orange} & \cellcolor{yellow} & \cellcolor{orange} & \cellcolor{yellow} & \cellcolor{orange}\\
                                \hline
                                $6$ & \cellcolor{red} & \cellcolor{white} & \cellcolor{red} & \cellcolor{white} & \cellcolor{red} & \cellcolor{white}\\
                            \end{tabular}
                        \end{center}
                    \end{table}
                    De los 18 casos en donde el segundo dado es impar, sólo en 3 de esos casos la suma es 7
                    \begin{equation*}
                        \begin{aligned}
                            {P(S|B)} &= \frac{P(S\cap B)}{P(B)}\\
                                     &= \frac{3}{18}\\
                        \end{aligned}
                    \end{equation*}
                    \item
                    \item 
                \end{enumerate}
        \end{enumerate}
    \end{examples*}
\end{properties}

\begin{definition}[Partición]
    Decimos que los eventos $B_{1}, B_{2}, \dots, B_{k}$ forman una partición de $\Omega$ si 
    \begin{enumerate}
        \item $B_{i} \cap B_{j} = \O$, $\forall i \neq j$.
        \item $\bigcup_{i=1}^{k} B_{i} = \Omega$
    \end{enumerate}
\end{definition}

\begin{defexamples}
    \begin{itemize}
        \item Si tomo una sección de vidrio cuadrada y la rompo, el vidrio quedó particionado. Cada uno de los pedazos de vidrio es un $B_{j}$ de mi partición. La intersección es vacía, pero si los uno contruyo mi $\Omega$ es decir el vidrio completo.
       \item Una pared formada por azulejos. Su intersección es nula, pero su unión es toda la pared. Si salpico salsa en la parted, veo una mancha distribuida en varios azulejos, es decir en parios pedazos de mi partición. Si la salsa es el conjuno $A$, lo voy a escribir como la unión de todos los azulejos en donde $A$ se interseca con cada $B_{i}$ es decir
       \begin{equation*}
           A = (A \cap B_{1}) \cup (A \cap B_{2}) \cup \dots \cup (A \cap B_{k})
       \end{equation*}
       Todos los eventos entre paréntesis son mutuamente exclutentes. Si quiero calcular $P(A)$ usando el axioma 3
       \begin{equation*}
           \begin{aligned}
               P(A) &= P((A \cap B_{1}) \cup (A \cap B_{2}) \cup \dots \cup (A \cap B_{k}))\\
                    &= P((A \cap B_{1})) + P((A \cap B_{2})) + \dots + P((A \cap B_{k}))
           \end{aligned}
       \end{equation*}
    \end{itemize}
\end{defexamples}

De este último ejemplo surge una definición

\begin{definition}Fórmula de Probabilidad Total\\
    Sea
    \begin{equation*}
        \begin{aligned}
            P(A) &= P((A \cap B_{1}) \cup (A \cap B_{2}) \cup \dots \cup (A \cap B_{k}))\\
                 &= P((A \cap B_{1})) + P((A \cap B_{2})) + \dots + P((A \cap B_{k}))
        \end{aligned}
    \end{equation*}
    sabiendo que 
    \begin{equation*}
        P(A \cap B) = P(A|B) \cdot P(B)
    \end{equation*}
    definimos a la fórmula de probabilidad total como
    \begin{equation*}
        P(A) = \sum_{i=1}^{k} P(A|B_{i}) \cdot P(B_{i})
    \end{equation*}
\end{definition}

\begin{defexample}
    Sé que la probabilidad de que un dado fabricado por una máquina M1 sea defectuoso es de 0.1, si es de la máquina 2 es de 0.05, y si es de la máquina 3 es de 0.01. La producción se divide como
    \begin{center}
        $M1 \rightarrow 20\%$ de la producción\\
        $M2 \rightarrow 30\%$ de la producción\\
        $M3 \rightarrow 50\%$ de la producción
    \end{center}
    Suponiendo que todos los dados fabricados van a parar a la misma caja. Si elijo un dado al azar de la caja, cuál es la probabilidad de que ese dado sea defectuoso'.

    Solución: Voy a leer de nuevo mi experimento y entender cual es el experimento aleatorio y a partir de el definir eventos. El EA consiste en "Elegir un dado de la caja al azar y observar si es defectuoso". Los eventos que me van a interesar definir son:

    \begin{center}
        $M_{i}$: "El dado elegido al azar proviene de la máquina $i$". $i=1, 2, 3$\\
        D: "El dado elegido es defectuoso".
    \end{center}

    Ahora intento visualizar todos los posibles resultados de mi experimento. Utilizando un diagrama de árbol:

    \begin{center}
        \subfile{../../diagrams/example_prob_tree.tex}
    \end{center}

    Tenemos que calcular $P(D)$, observamos en el árbol de probabilidades que la probabilidad de que el dado sea defectuoso se puede calcular como la probabilidad de que el dado sea defectuoso de la máquina 1, o que sea defectuoso de la máquina 2, o que sea defectuoso de la máquina 3:
    \begin{equation*}
        D = (D \cap M_{1}) \cup (D \cap M_{2}) \cup (D \cap M_{3})
    \end{equation*}
    Como las ramas son excluyentes (el dado no puede ser de más de una máquina al mismo tiempo), podemos calcular la probabilidad de $D$ de la siguiente manera
    \begin{equation*}
        \begin{aligned}
            P(D) &= P((D \cap M_{1}) \cup (D \cap M_{2}) \cup (D \cap M_{3}))\\
                 &= P(D \cap M_{1}) + P(D \cap M_{2}) + P(D \cap M_{3}))\\
                 &= P(M_{1}) \cdot P(D \cap M_{1}) + P(M_{2}) \cdot P(D \cap M_{2}) + P(M_{3}) \cdot P(D \cap M_{3})\\
                 &= 0.2 \cdot 0.1 + 0.3 \cdot 0.05 + 0.5 \cdot 0.01\\
                 &= 0.04
        \end{aligned}
    \end{equation*}
\end{defexample}

\begin{observation}
    La probabilidad obtenida no puede ser más chica que la producida por la "mejor" máquina, ni más grande que la producida por la "peor" máquina.
\end{observation}

\begin{defexample}
    Usando el enunciado y datos del ejemplo anterior, ahora se sabe que el dado es defectuoso, y quiero saber que máquina lo fabricó.

    Solución: Recordando que $P(M_{1}) = 0.2$ yo lo que quiero ver ahora es cómo se ve modificada esa probabilidad ahora que sé que el dado es defectuoso
    \begin{equation*}
        \begin{aligned}
            P(M_{1}|D) &= \frac{P(D \cap M_{1})}{P(D)}\\
                       &= \frac{P(D|M_{1}) \cdot P(M_{1})}{P(D)}\\
                       &= \frac{0.1 \cdot 0.2}{0.4}\\
                       &= 0.5
        \end{aligned}
    \end{equation*}
\end{defexample}

\begin{theorem}[Teorema de Bayes]
    Sean $B_{1}, \dots, B_{k}$ es una partición de $\Omega$, $A$ un evento de probabilidad positiva, entonces
    \begin{equation*}
        P(B_{i}|A) = \frac{P(A|B_{i}) \cdot P(B_{i})}{\sum_{j=1}^{k}P(A|B_{j})\cdot P(B_{j})}
    \end{equation*}
\end{theorem}

\begin{defexample}
    En una urna hay una bola verde y dos bolas rojas. En cada paso se extrae una bola al azar y se la repone junto con otra del mismo color.
    \begin{enumerate}[label=(\alph*)]
        \item Calcular la proabilidad de que al finalziar el segundo paso la urna contenga dos bolas verdes y 3 rojas.
        \item Si al finalizar el segundo paso la urna contiene dos bolas verdes y 3 rojas, ¿cuál es la probabilidad de que en el primer paso se haya extraido una bola roja?
    \end{enumerate}

    Solución: 
    \begin{enumerate}[label=(\alph*)]
        \item Voy a extraer una bolita, ver el color, y luego la voy a reponer una bola del mismo color. Me va a interesar cada vez que saco una bolita saber de qué color salio y además saber en qué paso la estoy realizando. Defino dos eventos
        \begin{center}
            $R_{i}$: "La extracción $i$ es roja".\\
            $V_{i}$: "La extracción $i$ es verde". $i = 1, 2$
        \end{center}
    
        \begin{center}
            \subfile{../../diagrams/bayes_theorem_example_prob_tree.tex}
        \end{center}
    
        Habiendo llenado mi árbol de probabilidad sabiendo que se trata de un espacio equiprobable, ahora defino un evento
    
        \begin{center}
            S: "Al finalizar, la urna contiene $2V, 3R$".
        \end{center}
    
        Tengo 2 ramas que corresponden al evento, es decir, dos caminos posibles que me llevan a ese evento.
    
        Aplicando el teorema de Probabilidad Total
    
        \begin{equation*}
            \begin{aligned}
                P(S) &= P((R_{1} \cap V_{2}) \cup (V_{1} \cap R_{2}))\\
                     &= P(R_{1}) \cdot P(V_{2}|R_{1}) + P(V_{1}) \cdot P(R_{2}|V_{1})\\
                     &= \frac{2}{5} \cdot \frac{1}{4} + \frac{1}{5} \cdot \frac{1}{2}\\
                     &= \frac{1}{3}
            \end{aligned}
        \end{equation*}
        \item Nos están pidiendo es que de todas las opciones posibles que teníamos, me quedo con las que acabamos de analizar.
            \begin{equation*}
                \begin{aligned}
                    P(R_{1}|S) &= \frac{P(R_{1} \cap S)}{P(S)} \therefore S = (R_{1} \cap V_{2}) \cup (V_{1} \cap R_{2}) \Rightarrow S \cap R_{1} = R_{1} \cap V_{2}\\
                               &= \frac{P(V_{2}|R_{1}) \cdot P(R_{1})}{P(S)}\\
                               &= \frac{\frac{1}{4} \cdot \frac{2}{3}}{\frac{1}{3}}\\
                               &= \frac{1}{2}
                \end{aligned}
            \end{equation*}
    \end{enumerate}
    
\end{defexample}