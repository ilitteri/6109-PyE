\documentclass[../../main.tex]{subfiles}

Se trata de analizar cómo afecta la información de que "un evento B ha ocurrido" a la probabilidad asignada de A.

\begin{definition}[Probabilidad Condicional]
    Sea $(\Omega, \mathcal{A}, P)$ un espacio de probabilidad, $A, B \in \mathcal{A}$ con $P(B)> 0$, la probabilidad condicional de $A$ dado que $B$ ha ocurrido está definida por:
        \begin{equation*}
            P(A|B) = \frac{P(A \cap B)}{B}
        \end{equation*}
\end{definition}

\begin{properties}
    La $P(A|B)$ para un $B$ fijo satisface todos los axiomas de probabilidad:
    \begin{enumerate}
        \item $0 \leq P(A|B) \leq 1$, $\forall A \in \mathcal{A}$
            \begin{proof}
                $$P(A|B) = \frac{P(A \cap B)}{B} \leq 1$$
                $$\text{Si } A \cap B = \O \rightarrow P(A \cap B) = 0$$
                $$\begin{aligned}
                    \text{Si } A \cap B \neq \O &\rightarrow A \cap B \subseteq B\\
                    &\rightarrow P(A \cap B) \leq P(B)
                \end{aligned}$$
            \end{proof}
        \item $P(\Omega|B) = 1$
            \begin{proof}
                $$P(\Omega|B) = \frac{\Omega \cap B}{P(B)} = \frac{P(B)}{P(B)} \because B \subseteq \Omega$$
            \end{proof}
        \item Si $A \cap C = \O \rightarrow P(A \cup C|B) = P(A|B) + P(C|B)$
        \item Si $P(B) > 0$
            \begin{enumerate}
                \item $P(A \cap B) = P(A|B) \cdot P(B) = P(B|A) \cdot P(A)$.
                \item $P(A \cap B \cap C) = P(A|B \cap C) \cdot P(B \cap B) = P(A|B \cap C) \cdot P(B|C) \cdot P(C)$
            \end{enumerate}
    \end{enumerate}

    \begin{examples*}
        \begin{enumerate}
            \item Si un programador usa C ¿cuál es la probabilidad de que use Java?\\
                Tenemos que $P(C) > 0$, y que lo que sabes es $P(C)$, y buscamos $P(J)$ entonces
                \begin{equation*}
                    P(J|C) = \frac{P(J \cap C)}{P(C)} = \frac{0 \cdot 1}{0 \cdot 4} = \frac{1}{4}
                \end{equation*}
            \item Si un programador no usa Java, ¿cuál es la probabilidad de que use C?\\
                Nuevamente identificamos que lo que me está pidiendo es una probabilidad condicional. Con la información que tenemos
                \begin{equation*}
                    P(C|\overline{J}) = \frac{P(C \cap \overline{J})}{P(\overline{J})} = \frac{0 \cdot 3}{0 \cdot 5} = \frac{3}{5}
                \end{equation*}
            \item Una persona arroja 2 dados equilibrados. Calcular la probabilidad de que la suma sea 7 dado que:
                \begin{enumerate}
                    \item La suma es impar.
                    \item La suma es mayor que 6.
                    \item El número del $2^{do}$ dado es impar.
                    \item El número de alguno de los dados es impar.
                    \item Los dos números son iguales.
                \end{enumerate}
                Defino mi EA: "Arrojo dos dados y observo". Ahora defino un evento $D_{i}$: "Valor observado en el dado $i$", $i= 1, 2$.
                \begin{enumerate}[label=(\alph*)]
                    \item Debo calcular la probabilidad de que la suma sea 7 dado que la suma es impar, dicho de otra manera, yo ya se que la suma es impar, y a partir de ello quiero calcular la probabilidad de que la suma sea 7.
                    \begin{table}[H]
                        \begin{center}
                            \begin{tabular}{c|c|c|c|c|c|c}
                                $D_{2}/D_{1}$ & $1$ & $2$ & $3$ & $4$ & $5$ & $6$\\
                                \hline
                                $1$ & \cellcolor{white} & \cellcolor{red} & \cellcolor{white} & \cellcolor{red} & \cellcolor{white} & \cellcolor{orange}\\
                                \hline
                                $2$ & \cellcolor{red} & \cellcolor{white} & \cellcolor{red} & \cellcolor{white} & \cellcolor{orange} & \cellcolor{white}\\
                                \hline
                                $3$ & \cellcolor{white} & \cellcolor{red} & \cellcolor{white} & \cellcolor{orange} & \cellcolor{white} & \cellcolor{red}\\
                                \hline
                                $4$ & \cellcolor{red} & \cellcolor{white} & \cellcolor{orange} & \cellcolor{white} & \cellcolor{red} & \cellcolor{white}\\
                                \hline
                                $5$ & \cellcolor{white} & \cellcolor{orange} & \cellcolor{white} & \cellcolor{red} & \cellcolor{white} & \cellcolor{red}\\
                                \hline
                                $6$ & \cellcolor{orange} & \cellcolor{white} & \cellcolor{red} & \cellcolor{white} & \cellcolor{red} & \cellcolor{white}\\
                            \end{tabular}
                        \end{center}
                    \end{table}
                    \begin{equation*}
                        P(\underbrace{D_{1}+D_{2} = 7}_{\colorbox{yellow}{S}}|\underbrace{\text{"La suma es impar"}}_{\colorbox{red}{A}})
                    \end{equation*}
                    \begin{equation*}
                        \begin{aligned}
                            {P(S|A)} &= \frac{P(S\cap A)}{P(A)}\\
                                     &= \frac{\frac{6}{36}}{\frac{18}{36}}\\
                                     &= \frac{6}{18}
                        \end{aligned}
                    \end{equation*}
                    Este resultado se puede interpretar como, de los 18 cuadraditos verdes, 6 son naranjas. En otras palabras, cuando tengo un espacio equiprobable y condiciono, es decir, quito posbiles resultados de mi experimento, los resultados restantes siguen siendo equiprobables, lo que me permite seguir calculando la probabilidad con Laplace. Esto hace que en espacios equiprobables hacer las cuentas sea mucho mas fácil.
                    \item \colorbox{cyan}{B}: "La suma es mayor que 6"
                    \begin{table}[H]
                        \begin{center}
                            \begin{tabular}{c|c|c|c|c|c|c}
                                $D_{2}/D_{1}$ & $1$ & $2$ & $3$ & $4$ & $5$ & $6$\\
                                \hline
                                $1$ & \cellcolor{white} & \cellcolor{red} & \cellcolor{white} & \cellcolor{red} & \cellcolor{white} & \cellcolor{violet}\\
                                \hline
                                $2$ & \cellcolor{red} & \cellcolor{white} & \cellcolor{red} & \cellcolor{white} & \cellcolor{violet} & \cellcolor{cyan}\\
                                \hline
                                $3$ & \cellcolor{white} & \cellcolor{red} & \cellcolor{white} & \cellcolor{violet} & \cellcolor{cyan} & \cellcolor{cyan}\\
                                \hline
                                $4$ & \cellcolor{red} & \cellcolor{white} & \cellcolor{violet} & \cellcolor{cyan} & \cellcolor{cyan} & \cellcolor{cyan}\\
                                \hline
                                $5$ & \cellcolor{white} & \cellcolor{violet} & \cellcolor{cyan} & \cellcolor{cyan} & \cellcolor{cyan} & \cellcolor{cyan}\\
                                \hline
                                $6$ & \cellcolor{violet} & \cellcolor{cyan} & \cellcolor{cyan} & \cellcolor{cyan} & \cellcolor{cyan} & \cellcolor{cyan}\\
                            \end{tabular}
                        \end{center}
                    \end{table}
                    De los 21 casos en donde la suma es mayor a 6, sólo en 6 la suma es 7
                    \begin{equation*}
                        \begin{aligned}
                            {P(S|B)} &= \frac{P(S\cap B)}{P(B)}\\
                                     &= \frac{6}{21}\\
                        \end{aligned}
                    \end{equation*}
                    \item \colorbox{yellow}{C}: "El $2^{do}$ dado es impar"
                    \begin{table}[H]
                        \begin{center}
                            \begin{tabular}{c|c|c|c|c|c|c}
                                $D_{2}/D_{1}$ & $1$ & $2$ & $3$ & $4$ & $5$ & $6$\\
                                \hline
                                $1$ & \cellcolor{yellow} & \cellcolor{orange} & \cellcolor{yellow} & \cellcolor{orange} & \cellcolor{yellow} & \cellcolor{orange}\\
                                \hline
                                $2$ & \cellcolor{red} & \cellcolor{white} & \cellcolor{red} & \cellcolor{white} & \cellcolor{red} & \cellcolor{white}\\
                                \hline
                                $3$ & \cellcolor{yellow} & \cellcolor{orange} & \cellcolor{yellow} & \cellcolor{orange} & \cellcolor{yellow} & \cellcolor{orange}\\
                                \hline
                                $4$ & \cellcolor{red} & \cellcolor{white} & \cellcolor{red} & \cellcolor{white} & \cellcolor{red} & \cellcolor{white}\\
                                \hline
                                $5$ & \cellcolor{yellow} & \cellcolor{orange} & \cellcolor{yellow} & \cellcolor{orange} & \cellcolor{yellow} & \cellcolor{orange}\\
                                \hline
                                $6$ & \cellcolor{red} & \cellcolor{white} & \cellcolor{red} & \cellcolor{white} & \cellcolor{red} & \cellcolor{white}\\
                            \end{tabular}
                        \end{center}
                    \end{table}
                    De los 18 casos en donde el segundo dado es impar, sólo en 3 de esos casos la suma es 7
                    \begin{equation*}
                        \begin{aligned}
                            {P(S|B)} &= \frac{P(S\cap B)}{P(B)}\\
                                     &= \frac{3}{18}\\
                        \end{aligned}
                    \end{equation*}
                    \item
                    \item 
                \end{enumerate}
        \end{enumerate}
    \end{examples*}
\end{properties}


\subsubsection{Particion}
\subfile{subsubsections/particion.tex}

De este último ejemplo surge una definición

\subsubsection{Funcion de Probabilidad Total}
\subfile{subsubsections/funcion_de_probabilidad_total.tex}

\subsubsection{Teorema de Bayes}
\subfile{subsubsections/teorema_de_bayes.tex}