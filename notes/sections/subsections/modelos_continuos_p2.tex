\documentclass[../../main.tex]{subfiles}

\subsubsection{Distribucion Uniforme}
\subfile{subsubsections/distribucion_uniforme.tex}

\subsubsection{Distribucion Exponencial}
\subfile{subsubsections/distribucion_exponencial.tex}


\begin{definition}[Funcion de Riesgo (para una $V.A.C$)]
    Pregunta: Dado que un compente duro un cierto tiempo $t$, cual es la probabilidad de que se rompa un instante despues?
    \begin{center}
        T: "Tiempo hasta que el componente falla"
    \end{center}
    \begin{equation*}
        P(T<t+\Delta{t}|T>t)
    \end{equation*}
    Esta probabilidad se puede pensar como una funcion de $t$ a la que vamos a llamar funcion intensidad de fallas multiplicado por un intervalo $\Delta{t}$
    \begin{equation*}
        \begin{aligned}
            \lambda(t) &= \lim_{\Delta{t}\rightarrow0} \frac{P(T<t+\Delta{t}|T>t)}{\Delta{t}}\\
                       &= \lim_{\Delta{t}\rightarrow0} \frac{P(t < T < t+\Delta{t})}{\Delta{t} \cdot P(T>t)}\\
                       &= \lim_{\Delta{t}\rightarrow0} \frac{F_T(t + \Delta{t}) - F_T(t)}{\Delta{t} \cdot (1 - F_T(t))}\\
                       &= \frac{f_T(t)}{1 - F_T(t)}
        \end{aligned}
    \end{equation*}
    Este ultimo cociente se parece  a la derivada del logaritmo. Porque la funcion de densidad es la derivada de la funcion de distribucion, entonces
    \begin{equation*}
        -\lambda(t) = \frac{\partial}{\partial{t}} \ln{1 - F_T(t)}
    \end{equation*}
    \begin{equation*}
        -\int_{0}^{t}\lambda(t) ds= \ln{1 - F_T(t)}
    \end{equation*}
    \begin{equation*}
        F_{T}(t) = 1 - e^{-\int_{0}^{t}\lambda(t) ds}
    \end{equation*}
\end{definition}
Cuando nos den la funcion intensidad de falla vamos a poder encontrar la funcion de distribucion de nuestra variable aleatoria que en general va a medir un tiempo hasta un evento, usando la formula a la que acabamos de lllegar, y con esa funcion de distribucion podemos calcular cualquier cosa que nos pidan.

\subsubsection{Distribucion Gamma}
\subfile{subsubsections/distribucion_gamma.tex}

\subsubsection{Distribucion Normal Estandar}
\subfile{subsubsections/distribucion_normal_estandar.tex}

\subsubsection{Cuantil}
\subfile{subsubsections/cuantil.tex}