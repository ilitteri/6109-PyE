\documentclass[../../main.tex]{subfiles}

\begin{definition}[Simulación]
    "Imitar o fingir que se está realizando una acción cuando en realidad no se está llevando a cabo."
\end{definition}

\subsubsection{¿Cómo simulo el tirar un dado?}
\subfile{../../diagrams/simulacion_dado.tex}

Tengo que definir una regla

\begin{center}
    x: Valor al elegir un n° al azar entre 0 y 1\\
    D: Valor observado en el dado
\end{center}
\begin{equation*}
    \begin{aligned}
        \text{Si } 0 \leq x \leq \frac{1}{6} \rightarrow \text{D = 1}\\
        \text{Si } \frac{1}{6} \leq x \leq \frac{2}{6} \rightarrow \text{D = 2}\\
        \text{Si } \frac{2}{6} \leq x \leq \frac{3}{6} \rightarrow \text{D = 3}\\
        \text{Si } \frac{3}{6} \leq x \leq \frac{4}{6} \rightarrow \text{D = 4}\\
        \text{Si } \frac{4}{6} \leq x \leq \frac{5}{6} \rightarrow \text{D = 5}\\
        \text{Si } \frac{5}{6} \leq x \leq \frac{6}{6} \rightarrow \text{D = 6}\\
    \end{aligned}
\end{equation*}

Para estimar probabilidades a partir de simulaciones, vamos a usar la idea de frecuencia relativa, vamos a "tirar el dado" muchas veces, y teniendo una idea de chances contando la cantidad de veces que ocurría el evento que queremos observar y lo dividimos por la cantidad de veces que realizo el experimento.

