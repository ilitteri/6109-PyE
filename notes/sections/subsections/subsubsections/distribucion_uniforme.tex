\documentclass[../../../main.tex]{subfiles}

\begin{definition}[Distribucion Uniforme]
    Supongamos que $X$ es una $V.A.C$ que toma todos los valores sobre el intervalo $[a, b]$. Si $f_{X}(x)$ esta dada por
    
    \begin{equation*}
        f_X(x) = 
        \left\{
            \begin{aligned}
                &k = \frac{1}{b-a} \text{ si }a < x < b\\
                &0 \text{ e.o.c}
            \end{aligned}
        \right.
    \end{equation*}
    
    Estamos hablando de una variable continua por lo tanto tiene funcion de densidad y se llama uniforme. Cuando hablamos de una variable uniforme lo que estamos diciendo es que su funcion de densidad va a ser constante sobre el intervalo $[a, b]$ y 0 en otros casos, de manera que todos los valores que se encuentren en el intervalo tienen la misma probabilidad de ocurrir. Si graficamos
    
    \begin{center}
        \subfile{../../../diagrams/distribucion_uniforme_explicacion.tex}
    \end{center}
    
    Entonces, para que sea una funcion de densidad
    
    \begin{enumerate}
        \item $k > 0$
        \item $\int_{-\infty}^{\infty}f_X(x) dx = 1$
    \end{enumerate}
    
    Se dice que la varaibla aleatoria X tiene disstribucion uniforme de paramteros a y b
    
    \begin{equation*}
        X \thicksim \mathcal{U}(a, b)
    \end{equation*}
    
    Esta distribucion nos debe venir a la cabeza cuando se hable de elegir un punto al azar sobre un continuo.
\end{definition}


\begin{defexample}
    Una vara de longitud $10m$ se corta en un punto elegido al azar. Calcular la probabilidad de que la pueza mas corta mida menos de $3m$.

    Solución: lo primero que vamos a hacer es identificar y definir a nuestra variable aleatoria. Identificando que lo aleatorio en mi experimento es el corte de barra, mi variable aleatoria es el punto en el que se corta la barra
    \begin{center}
        X: "punto den el que se corta la vara".
    \end{center}
    Estamos eligiendo al azar en un continuo, entonces nuestra variable aleatoria tiene una distribucion continua en un intervalor dado
    \begin{equation*}
        X \thicksim \mathcal{U}(0, 10)
    \end{equation*}
    Luego
    \begin{equation*}
        P(\text{"Pieza mas corta mida menos de 3m})
    \end{equation*}
    \begin{equation*}
        P(\underbrace{(X < 3, X < 5)}_{X<3} \cup \underbrace{(10-X < 3, X > 5)}_{X>7})
    \end{equation*}
    \begin{center}
        \subfile{../../../diagrams/distribucion_uniforme_ejemplo.tex}
    \end{center}
    Nos quedaron areas de rectangulos entonces
    \begin{equation*}
        \begin{aligned}
            P(X <3, X>7) &= \frac{3}{10} + \frac{3}{10}\\
                         &= \frac{6}{10}
        \end{aligned}
    \end{equation*}
\end{defexample}