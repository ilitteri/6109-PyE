\documentclass[../../../main.tex]{subfiles}

\begin{definition}[Esperanza total]
    Si despejo y pienso en una partición tenemos
    \begin{equation*}
        E(X) = E(X|X \in A) \cdot P(X \in A) + E(X|X \in \overline{A}) \cdot P(X \in \overline{A})
    \end{equation*}
    que es el cálculo de esperanza total
\end{definition}

\begin{example*}
    Hallar $E(T|T<1)$ cuadno $T \sim \mathcal{E}(1)$.\\
    Solución: sabemos de la distribución exponencial que
    \begin{itemize}
        \item $E(T) = 1$.
        \item $T|T>1 = 1 + T'$ donde $T' \sim \mathcal{E}(1)$.
        \item $T$ no tiene memoria.
    \end{itemize}
    Luego
    \begin{equation*}
        \begin{aligned}
            E(T|T>1) &= 1 + E(T)\\
                     &= 2 
        \end{aligned}
    \end{equation*}
    Usando la fórmula de esperanza total
    \begin{equation*}
        E(T) = E(T|T<1) \cdot P(T<1) + E(T|T\geq1) \cdot P(T\geq1)
    \end{equation*}
    despejando
    \begin{equation*}
        E(T|T<1) = 0.418
    \end{equation*}
\end{example*}

\begin{property}
    Si yo grafico una función de densidad cualquiera, la parte de $+x$ voy a calcular el área por encima de la función de distribución, y a eso le voy a restar el área por debajo de la función para todos los $-x$.
    \begin{equation*}
        E(X) = \int_{0}^{\infty} (1 - F_{X}(x)) dx - \int_{-\infty}^{0} F){X}(x) dx
    \end{equation*}
\end{property}

\begin{example*}
    Hallar la esperanza de $T* = min(T, 1)$ con $T \sim \mathcal{E}(1)$
    \begin{equation*}
        T* = 
        \left\{
            \begin{aligned}
                &T &T<1\\
                &1 &T\geq1
            \end{aligned}
        \right.
    \end{equation*}
    \begin{equation*}
        \begin{aligned}
            E(T*) &= \int_{0}^{1} 1 - (1 - e^{-t}) dt\\
                  &= \int_{0}^{1}e^{-t} dt\\
                  &= 1 - e^{-1}
        \end{aligned}
    \end{equation*}
    Si usamos la definición de $E(h(X))$, como $T$ es continua me queda una integral por partes.
\end{example*}