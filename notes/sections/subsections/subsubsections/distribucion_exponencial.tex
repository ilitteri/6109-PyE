\documentclass[../../../main.tex]{subfiles}

\begin{definition}[Distribucion Exponencial]
    Una variable aleatoria tiene distribucion exponencial de parametro $\lambda > 0$ si su funcion de densidad esta dada por
    \begin{equation*}
        f_X(x) = 
        \left\{
            \begin{aligned}
                &\lambda e^{-\lambda x} \text{ si }a < x < b\\
                &0 \text{ e.o.c}
            \end{aligned}
        \right.
    \end{equation*}

    \begin{center}
        \subfile{../../../diagrams/distribucion_exponencial.tex}
    \end{center}

    \begin{equation*}
        \begin{aligned}
            \int_{-\infty}^{\infty}f_X(x) dx &= \int_{0}^{\infty} \lambda e^{-\lambda x} dx\\
                                             &= -e^{-\lambda x}\Biggr|_{0}^{\infty}\\
                                             &= 1
        \end{aligned}
        \text{siempre que } \lambda > 0, \lambda e^{-\lambda x}>0
    \end{equation*}
\end{definition}

Calculando su funcion de distribucion

\begin{equation*}
    F_X(x) = P(X \leq x) =
    \left\{
        \begin{aligned}
            &0 &x < 0\\
            &\int_{0}^{x} \lambda e^{-\lambda t} dt = 1 - e^{-\lambda t} &x\geq 0
        \end{aligned}
    \right.
\end{equation*}

Entonces, i $x<0$

\begin{equation*}
    P(X>x) = 1 - F_{X}(x = e^{-\lambda x})
\end{equation*}

\begin{properties}
    \begin{enumerate}
        \item Si $X \sim \mathcal{E}(\lambda)$ entonces $P(X>t+s, X>t) = P(X > s)$, $\forall t, s \in \mathbb{R}^{+}$
        \begin{proof}
            \begin{equation*}
                \begin{aligned}
                    P(X>t+s|X>t) &= \frac{P(X>t+s, X>t)}{P(X>t)}\\
                                 &= \frac{e^{-\lambda (t+s)}}{e^{-\lambda t}}\\
                                 &= e^{-\lambda s}\\
                                 &= P(X>s)
                \end{aligned}
            \end{equation*}
        \end{proof}
        \textbf{Ejemplo}\\
        Supongamos que X es la duracion de una lampara de bajo consumo, entonces, la ecuacion anterior expresaria, que probabilidad hay de que mi lampara dure 8 horas si ya se que duro 4 horas. Y desarrollando llegamos a que esa expresion es equivalente a decir que probabilidad hay de que mi lampara dure mas de 4 horas. Esta propiedad se llama perdida de memoria.
        \item Si $X$ es una $V.A.C$ y $P(X>t+s|X>t) = P(X>s)$, $\forall t, s \in \mathbb{R}^{+}$ entonces existe $\lambda > 0$ tal que $X \sim \mathcal{E}(\lambda)$. En otras palabras, si $X$ es continua y tiene perdida de memoria entonces $X$ es una distribucion exponencial
    \end{enumerate}
\end{properties}