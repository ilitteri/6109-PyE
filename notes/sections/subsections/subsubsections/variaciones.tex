\documentclass[../../../main.tex]{subfiles}

\begin{definition}[Variaciones]
    Es la cantidad de subconjuntos ordenados de r elementos que se pueden formar a partir de un conjunto de n elementos. Las variaciones son, de un conjunto total de n elementos son todos los subconjuntos diferentes que puedo extraer si me importa el orden en el que los estoy extrayendo.
    \begin{equation*}
        P_{n,r} = \frac{n!}{(n-r)!}
    \end{equation*}
    \begin{equation*}
        \text{En la calculadora: } \boxed{nPr}
    \end{equation*}
\end{definition}

\begin{defexample}
    En un cine con 70 butacas, ¿de cuántas formas distintas pueden sentarse 45 personas?
    \begin{equation*}
        \#CP = \frac{70!}{(70-45)!} = \frac{70!}{25!}
    \end{equation*}
\end{defexample}