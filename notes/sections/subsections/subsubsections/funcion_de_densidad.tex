\documentclass[../../../main.tex]{subfiles}

\begin{definition}[Funcion de Densidad de Probabilidad]
    Se dice que $X$ es una $V.A.C.$ si existe una funcion $f_{X}: \mathbb{R} \rightarrow \mathbb{R}$, llamada \textbf{funcion de densidad de probabilidad}, que satisface las siguientes condiciones
    \begin{enumerate}
        \item $f_{X} \geq 0$, $\forall{x} \in \mathbb{R}$
        \item $\int_{-\infty}^{\infty}f_{X}(x)dx = 1$
        \item Para cualquier $a$ y $b$ tales que $-\infty < a < b < +\infty$ tenemos
        \begin{equation*}
            P(a < X < b) = \int_{a}^{b}f_{X}(x)dx
        \end{equation*}
    \end{enumerate}
    \begin{center}
        GRAFICO
    \end{center}
\end{definition}

\begin{defexample}
    La demanda de aceite pesado en cientos de litros durante una temporada tiene la siguiente funcion de densidad
    \begin{equation*}
        f_{X}(x) = \frac{4x + 1}{3} \cdot \mathbf{1}\{0<x<1\}
    \end{equation*}
    \begin{enumerate}
        \item Graficar $f_{X}(x)$ y verificar que sea una funcion de densidad.
        \item Hallar la funcion de distribucion de $X$.
        \item Calcular $P(\frac{1}{3} < X \leq \frac{2}{3})$ y $P(\frac{1}{3} < X \leq \frac{2}{3}|X < \frac{1}{2})$
    \end{enumerate}
    \begin{enumerate}
        \item 
        \textbf{Paso 1:} Defino una variable aleatoria.
        \begin{center}
            X: "Demanda de aceite en cientos de litros".
        \end{center}
        \textbf{Paso 2:} Grafico y verifico si es de densidad
        \begin{equation*}
            f_{X}(x) = 
            \left\{
                \begin{aligned}
                    &\frac{4x + 1}{3} &0<x<1\\
                    &0 &e.o.c.
                \end{aligned}
            \right.
        \end{equation*}
        \begin{enumerate}
            \item $f_{X} \geq 0$. SI
            \item $\int_{-\infty}^{\infty}f_{X}(x)dx = 1$. SI (calculas la integral o calcular el area de la figura bajo la curva, en este caso, el area del trapecio).
        \end{enumerate}
        \item 
        \begin{equation*}
            \begin{aligned}
                F_{X}(x) &= P(X \leq x)\\
                         &= \int_{-\infty}^{x}f_{X}(t)dt\\
                         &=
                         \left\{
                             \begin{aligned}
                                &\int_{-\infty}^{x}0dx = 0 &x< 0\\
                                &\int_{-\infty}^{0}0dx + \int_{0}^{x}\frac{4t + 1}{3}dt &0 \leq x < 1\\
                                &1 &x\geq 1
                             \end{aligned}
                         \right.\\
                         &= 
                         \left\{
                            \begin{aligned}
                               &0 &x< 0\\
                               &2x^{2} + x &0 \leq x < 1\\
                               &1 &x\geq 1
                            \end{aligned}
                         \right.\\
            \end{aligned}
        \end{equation*}
        Ahora graficamos la funcion de probabilidad para verificar si cumple con las condiciones.
    \end{enumerate}
\end{defexample}

\begin{note}
    Si $X$ es una $V.A.C.$, $F_{X}(x)$ es una funcion continua $\forall{x} \in \mathbb{R}$.
\end{note}

\begin{theorem}
    Sea $F_{X}(x)$ la funcion de distribucion de una $V.A.C.$ (admite derivada), luego
    \begin{equation*}
        f_{X}(x) = \frac{\partial}{\partial{x}} F_{X}(x)
    \end{equation*}
\end{theorem}

\begin{note}
    La funcion de densidad solo existe para $V.A.C.$
\end{note}