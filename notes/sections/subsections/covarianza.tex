\documentclass[../../main.tex]{subfiles}

\begin{definition}
    Sean $X$ e $Y$ dos variables aleatorias
    \begin{equation*}
        Cov(X, Y) = E((X - E(X)) \cdot (Y - E(Y)))
    \end{equation*}
\end{definition}

\begin{properties}
    \begin{enumerate}
        \item $Cov(X, Y) = E(X \cdot Y) - E(X) \cdot E(Y)$
        \begin{proof}
            \begin{equation*}
                \begin{aligned}
                    Cov(X, Y) &= E((X - E(X)) \cdot (Y - E(Y)))\\
                              &= E(X \cdot Y - x \cdot E(Y) - Y \cdot E(X) + E(X) \cdot E(Y))\\
                              &= E(X \cdot Y) - E(Y) \cdot E(X) - E(X) \cdot E(Y) + E(X) \cdot E(Y)\\
                              &= E(X \cdot Y) - E(X) \cdot E(Y)
                \end{aligned}
            \end{equation*}
        \end{proof}
        \item Si $X$ e $Y$ son independientes entonces la esperanza del producto entre ellas se puede calcular como el producto de sus esperanzas.
        \begin{equation*}
            E(X \cdot Y) = E(X) \cdot E(Y)
        \end{equation*}
        y por lo tanto
        \begin{equation*}
            Cov(X, Y) = 0
        \end{equation*}
    \end{enumerate}
    \item $Cov(a + bX, c +dY) = b \cdot d \cdot Cov(X, Y)$
    \begin{proof}
        \begin{equation*}
            \begin{aligned}
                foto 
            \end{aligned}
        \end{equation*}
    \end{proof}
    \item $Cov(X + Y, Z) = Cov(X, Z) + Cov(Y, Z)$.
    \begin{proof}
        
    \end{proof}
    \item $Var(X + Y) = Var(X) + Var(Y) + 2 \cdot Cov(X, Y)$.
    \begin{proof}
        \begin{equation*}
            \begin{aligned}
                foto
            \end{aligned}
        \end{equation*}
    \end{proof}
\end{properties}

\subsubsection{Coeficiente de Correlación}

\begin{definition}
    El coeficiente de correlación entre las variables aleatorias $X$ e $Y$ está dado por
    \begin{equation*}
        \rho_{X, Y} = \frac{Cov(X, Y)}{\sqrt{Var(X) \cdot Var(Y)}}
    \end{equation*}
    Lo que estamos haciendo es llevar a que el valor absoluto de la covarianza esté entre 0 y 1.
\end{definition}

\begin{property}
    \begin{equation*}
        |\rho_{X, Y}| = 1 \leftrightarrow P(aX + bY) = 1
    \end{equation*}
\end{property}

Si $\rho_{X, Y} = 0$ eso nos dice que no hay correlación lineal entre las variables, lo que puede significar que no hay dependencia o que hay correlación de otro tipo.