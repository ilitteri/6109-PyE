\documentclass[../../main.tex]{subfiles}

Dado un $E.A.$ y un $\Omega$ el espacio muestral asociado a el, una funcion $X$ que asigna a cada uno de los elementos $\omega \in \Omega$ un numero real $X(w)$ se llama variable aleatoria.

\begin{definition}[Variable Aleatoria]
    Sea $(\Omega, \mathcal{A}, P)$ un espacio de probabilidad y $X: \Omega \rightarrow \mathbb{R}$ una funcion, diremos que $X$ es una variable aleatoria si $X^{-1}(B) \in \mathcal{A}$.
\end{definition}

\begin{proposition}
    Sea $(\Omega, \mathcal{A}, P)$ un espacio de probabilidad y $X$ una $V.A.$ entonces $X^{-1}(B) \in \mathcal{A}$. Luego, se puede calcular la probabilidad, es decir
    \begin{equation*}
        P(X^{-1}(B)) = P(X \in B)
    \end{equation*}
\end{proposition}

\begin{observation}
    \begin{equation*}
        X^{-1}(B) = \{\omega \in \Omega: X(\omega) \in B\}
    \end{equation*}
\end{observation}

\subsubsection{Funcion de Distribucion}
\subfile{subsubsections/funcion_de_distribucion.tex}

\subsubsection{Variable Aleatoria Discreta}
\subfile{subsubsections/variable_aleatoria_discreta.tex}


\subsubsection{Funcion de Probabilidad}
\subfile{subsubsections/funcion_de_probabilidad.tex}

\begin{example*}
    De un conjunto de 7 ingenieras y 4 matematicas se eligen al azar 5 personas. Sea $X$ el numero de ingenieras elegidas
    \begin{enumerate}
        \item Hallar la funcion de probabilidad de $X$ y graficarla.
        \item Hallar la funcion de distribucion de $X$ y graficarla.
        \item Cual es la probabilidad de que el grupo elegido este formado por 3 ingenieras?
    \end{enumerate}
    \textbf{Paso 1:} defino la variable aleatoria
    \begin{center}
        X: "Cantidad de ingenieras en el grupo de 5".
    \end{center}
    \textbf{Paso 2:} encontrar los valores posibles de mi variable aleatoria
    \begin{equation*}
        R_{X} = \{1, 2, 3, 4, 5\}
    \end{equation*}
    \textbf{Paso 3:} encontrar la funcion de probabilidad
    \begin{equation*}
        p_{X}(x) = P(X = x) \forall{x} \in R_{X}
    \end{equation*}
    Empecemos: saco 5 personas al azar, sin importarme el orden
    \begin{equation*}
        \#CP = \binom{11}{5}
    \end{equation*}
    Ahora por Laplace calculo las funciones de probabilidad
    \begin{equation*}
        \begin{aligned}
            p_{X}(1) = P(X = 1) = \frac{\binom{7}{1}\binom{4}{4}}{\binom{11}{5}} = \frac{7}{462}\\
            p_{X}(2) = P(X = 2) = \frac{\binom{7}{2}\binom{4}{3}}{\binom{11}{5}} = \frac{84}{462}\\
            p_{X}(3) = P(X = 3) = \frac{\binom{7}{3}\binom{4}{2}}{\binom{11}{5}} = \frac{210}{462}\\
            p_{X}(4) = P(X = 4) = \frac{\binom{7}{4}\binom{4}{1}}{\binom{11}{5}} = \frac{140}{462}\\
            p_{X}(5) = P(X = 5) = \frac{\binom{7}{5}\binom{4}{0}}{\binom{11}{5}} = \frac{21}{462}\\
        \end{aligned}
    \end{equation*}
    Para terminar con lo anterior, hace falta verificar la definicion
    \begin{enumerate}
        \item $P(X = x_{i}) \geq 0$
        \item $\sum p_{X}(x_{i}) = 1$
    \end{enumerate}
    \begin{center}
        GRAFICO
    \end{center}
    Ahora busco la funcion de distribucion
    \begin{equation*}
        F_{X}(x) = P(X \leq x) = 
        \left\{
            \begin{aligned}
                &0 &x<1\\
                &\frac{7}{462} &1\leq x < 2\\
                &\frac{7}{462} + \frac{84}{462} &2\leq x < 3\\
                &\frac{7}{462} + \frac{84}{462} + \frac{210}{462} &3\leq x < 4\\
                &\frac{7}{462} + \frac{84}{462} + \frac{210}{462} + \frac{140}{462} &4\leq x < 5\\
                &1 &x \geq 5\\
            \end{aligned}
        \right.
    \end{equation*}
    Verifico la definicion y confirmo que la funcion es siempre continua a derecha, y esta definida para todo $x \in \mathbb{R}$.
    Por ultimo
    \begin{equation*}
        \begin{aligned}
            P(X \geq 3) &= 1 - P(X \leq 2)\\ 
                        &= 1 - F_{X}(2)\\
                        &= 1 - \frac{91}{462}\\
                        &= \frac{371}{462}\\
        \end{aligned}
    \end{equation*}
\end{example*}

\subsubsection{Distribucion de Bernoulli}
\subfile{subsubsections/distribucion_bernoulli.tex}

\subsubsection{Funcion Indicadora}
\subfile{subsubsections/funcion_indicadora.tex}

\textbf{IMPORTANTE} Para trabajar en la guia 2, el procedimiento es:
\begin{center}
    Identificar modelos $\rightarrow$ Tabla de Distribuciones
\end{center}
$V.A.D.$:
\begin{center}
    \begin{enumerate}
        \item Encuentro la $p_{X}(x)$
        \item Voy a la tabla y veo a quien pertenece.
    \end{enumerate}
\end{center}

\begin{example*}
    \begin{equation*}
        p_{X}(x) = \frac{\binom{7}{x}\binom{4}{5-x}}{\binom{11}{5}}, x = 1, 2, 3, 4, 5
    \end{equation*}
\end{example*}

\subsubsection{Variable Aleatoria Continua}
\subfile{subsubsections/variable_aleatoria_continua.tex}

\subsubsection{Funcion de Densidad de Probabilidad}
\subfile{subsubsections/funcion_de_densidad.tex}

\subsubsection{Tipo de Variable Segun su Funcion de Distribucion}

\subsubsection{Atomo}
\subfile{subsubsections/atomo.tex}

\subsubsection{Soporte}
\subfile{subsubsections/soporte.tex}

\begin{example*}
    Sea $X$ una $V.A.$ con funcion de distribucion
    \begin{equation*}
        \begin{aligned}
            F_{X}(x) &= P(X \leq x)\\
                     &= \frac{x}{4} \cdot \mathbb{1}\{0 \leq x < 1\} + \frac{1}{3} \cdot \mathbb{1}\{1 \leq x < 2\} + \frac{x}{6} \cdot \mathbb{1}\{x \geq 4\}
        \end{aligned}
    \end{equation*}
    \begin{enumerate}
        \item Hallar
        \begin{equation*}
            P(1 < X \leq 4), P(1 \leq X \leq 4), P(1 \leq X < 4), P(1 < X < 4) 
        \end{equation*}
        \item Sean $A = [1, 3.5]$, $B = (0.5, 3)$ calcular
        \begin{equation*}
            P(X \in A), P(X \in B|X \in A)
        \end{equation*}
    \end{enumerate}
\end{example*}