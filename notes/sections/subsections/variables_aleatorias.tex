\documentclass[../../main.tex]{subfiles}

Dado un $E.A.$ y un $\Omega$ el espacio muestral asociado a el, una funcion $X$ que asigna a cada uno de los elementos $\omega \in \Omega$ un numero real $X(w)$ se llama variable aleatoria.

\begin{definition}[Variable Aleatoria]
    Sea $(\Omega, \mathcal{A}, P)$ un espacio de probabilidad y $X: \Omega \rightarrow \mathbb{R}$ una funcion, diremos que $X$ es una variable aleatoria si $X^{-1}(B) \in \mathcal{A}$.
\end{definition}

\begin{proposition}
    Sea $(\Omega, \mathcal{A}, P)$ un espacio de probabilidad y $X$ una $V.A.$ entonces $X^{-1}(B) \in \mathcal{A}$. Luego, se puede calcular la probabilidad, es decir
    \begin{equation*}
        P(X^{-1}(B)) = P(X \in B)
    \end{equation*}
\end{proposition}

\begin{observation}
    \begin{equation*}
        X^{-1}(B) = \{\omega \in \Omega: X(\omega) \in B\}
    \end{equation*}
\end{observation}

\begin{definition}[Funcion de Distribucion]
    Sea $(\Omega, \mathcal{A}, P)$ un espacio de probabilidad y $X$ una $V.A.$, definimos su funcion de distribucion $F_{X}: \mathbb{R} \rightarrow [0, 1]$ dada por
    \begin{equation*}
        F_{X} = P(X \leq x), \forall{x} \in \mathbb{R}
    \end{equation*}
\end{definition}

\begin{properties}
    \begin{enumerate}
        \item $F_{X}(x) \in [0, 1]$ $\forall{x} \in \mathbb{R}$.
        \item $F_{X}(x)$ es monotona no decreciente.
        \item $F_{X}(x)$ es continua a derecha.
        \item $\lim_{x \rightarrow -\infty} F_{X}(x) = 0$ y $\lim_{x \rightarrow +\infty} F_{X}(x) = 1$
    \end{enumerate}
\end{properties}

\begin{definition}[Variables Aleatorias Discretas]
    Sea $(\Omega, \mathcal{A}, P)$ un espacio de probabilidad y $X$ una $V.A.$, diremos que $X$ es una $V.A.$ discreta cuando existe $A \in \mathbb{R}$ finito o infinito numerable tal que $p_{X}(A) = 1$ donde
    \begin{equation*}
        p_{X}(A) = P(X \in A)
    \end{equation*}
    Una $V.A.$ discreta es aquela cuyos valores posibles constituyen un conjunto finito o infinito numerable.\\
    El rango de una $V.A.D.$ es
    \begin{equation*}
        R_{X} = \{x \in \mathbb{R}: p_{X} > 0\}
    \end{equation*}
    Notacion:
    \begin{center}
        $X$ (mayuscula) $\rightarrow$ $V.A.$\\
        $x$ (minuscula) $\rightarrow$ resultados posibles de la $V.A.$
    \end{center}
\end{definition}

\begin{definition}
    Sea $X$ una $V.A.D.$, se llama funcion de probabilidad de $X$ a una funcion $p_(X): \mathbb{R} \rightarrow [0, 1]$ tal que $p_{X}(x) = P(X = x)$. Con cada resultado posible $x_{i}$ asociamos un numero $p_{X}(x_{i}) = P(X = x_{i})$ que debe cmumplir:
    \begin{enumerate}
        \item $p_{X}(x_{i}) \geq 0$ $\forall{i}$
        \item $\sum_{x \in \mathbb{R}} p_{X}(x) = 1$
    \end{enumerate}
\end{definition}

\begin{example*}
    $E.A.$: tiro una moneda 2 veces y observo el resultado. Defino C: observo cara.
    \begin{equation*}
        \Omega = \{(C, C), (C, \overline{C}), (\overline{C}, C), (\overline{C}, \overline{C})\}
    \end{equation*}
    \begin{enumerate}
        \item X: "Cantidad de caras observadas al tirar una moneda 2 veces".
        \item $R_{X} = \{\underbrace{0}_{x_1}, \underbrace{1}_{x_2}, \underbrace{2}_{x_3}\}$
        \item Busquemos $p_{X}(x) = P(X = x)$ para cada $x \in \mathbb{R}$
        \begin{equation*}
            \begin{aligned}
                p_{X}(0) &= P(X = 0) = P((\overline{C}, \overline{C})) = \frac{1}{4}\\
                p_{X}(1) &= P(X = 1) = P((C, \overline{C}) \cup (\overline{C}, C)) = \frac{1}{2}\\
                p_{X}(2) &= P(X = 2) = P((C,  C)) = \frac{1}{4}\\
            \end{aligned}
        \end{equation*}
    \end{enumerate}
    Entonces
    \begin{equation*}
        \begin{aligned}
            P(\text{"alguna cara"}) &= P(X \geq 1)\\
                                    &= P(X = 1) + P(X = 2)\\
                                    &= \frac{3}{4}
        \end{aligned} 
    \end{equation*}
\end{example*}

\begin{theorem}
    Sea $X$ una $V.A.D.$, entonces
    \begin{equation*}
        p_{X}(B) = \sum_{x_{i} \in B} p_{X}(x_{i})
    \end{equation*}
\end{theorem}

Para la funcion de distribucion del ejemplo tenemos que

\begin{equation*}
    F_{X}(x) = P(X \leq x) = 
    \left\{
        \begin{aligned}
            &0 &x<0\\
            &\frac{1}{4} &0\leq x \leq 1\\
            &\frac{3}{4} &1  \leq x \leq 2\\
            &1 &2 \leq x
        \end{aligned}
    \right.
\end{equation*}

\begin{example*}
    De un conjunto de 7 ingenieras y 4 matematicas se eligen al azar 5 personas. Sea $X$ el numero de ingenieras elegidas
    \begin{enumerate}
        \item Hallar la funcion de probabilidad de $X$ y graficarla.
        \item Hallar la funcion de distribucion de $X$ y graficarla.
        \item Cual es la probabilidad de que el grupo elegido este formado por 3 ingenieras?
    \end{enumerate}
    \textbf{Paso 1:} defino la variable aleatoria
    \begin{center}
        X: "Cantidad de ingenieras en el grupo de 5".
    \end{center}
    \textbf{Paso 2:} encontrar los valores posibles de mi variable aleatoria
    \begin{equation*}
        R_{X} = \{1, 2, 3, 4, 5\}
    \end{equation*}
    \textbf{Paso 3:} encontrar la funcion de probabilidad
    \begin{equation*}
        p_{X}(x) = P(X = x) \forall{x} \in R_{X}
    \end{equation*}
    Empecemos: saco 5 personas al azar, sin importarme el orden
    \begin{equation*}
        \#CP = \binom{11}{5}
    \end{equation*}
    Ahora por Laplace calculo las funciones de probabilidad
    \begin{equation*}
        \begin{aligned}
            p_{X}(1) = P(X = 1) = \frac{\binom{7}{1}\binom{4}{4}}{\binom{11}{5}} = \frac{7}{462}\\
            p_{X}(2) = P(X = 2) = \frac{\binom{7}{2}\binom{4}{3}}{\binom{11}{5}} = \frac{84}{462}\\
            p_{X}(3) = P(X = 3) = \frac{\binom{7}{3}\binom{4}{2}}{\binom{11}{5}} = \frac{210}{462}\\
            p_{X}(4) = P(X = 4) = \frac{\binom{7}{4}\binom{4}{1}}{\binom{11}{5}} = \frac{140}{462}\\
            p_{X}(5) = P(X = 5) = \frac{\binom{7}{5}\binom{4}{0}}{\binom{11}{5}} = \frac{21}{462}\\
        \end{aligned}
    \end{equation*}
    Para terminar con lo anterior, hace falta verificar la definicion
    \begin{enumerate}
        \item $P(X = x_{i}) \geq 0$
        \item $\sum p_{X}(x_{i}) = 1$
    \end{enumerate}
    \begin{center}
        GRAFICO
    \end{center}
    Ahora busco la funcion de distribucion
    \begin{equation*}
        F_{X}(x) = P(X \leq x) = 
        \left\{
            \begin{aligned}
                &0 &x<1\\
                &\frac{7}{462} &1\leq x < 2\\
                &\frac{7}{462} + \frac{84}{462} &2\leq x < 3\\
                &\frac{7}{462} + \frac{84}{462} + \frac{210}{462} &3\leq x < 4\\
                &\frac{7}{462} + \frac{84}{462} + \frac{210}{462} + \frac{140}{462} &4\leq x < 5\\
                &1 &x \geq 5\\
            \end{aligned}
        \right.
    \end{equation*}
    Verifico la definicion y confirmo que la funcion es siempre continua a derecha, y esta definida para todo $x \in \mathbb{R}$.
    Por ultimo
    \begin{equation*}
        \begin{aligned}
            P(X \geq 3) &= 1 - P(X \leq 2)\\ 
                        &= 1 - F_{X}(2)\\
                        &= 1 - \frac{91}{462}\\
                        &= \frac{371}{462}\\
        \end{aligned}
    \end{equation*}
\end{example*}

\begin{definition}[Modelos]
    Una $V.A.$ tiene distribucion de \textbf{Bernoulli} si sus valores posibles son 0 y 1 y le asigna $P(X = 1) = p$. Es decir, 
    \begin{equation*}
        X \text{ es } V.A.D., R_{X} = \{0, 1\} \text{ y } p_{X}(0) = 1 - p; p_{X}(1) = p.
    \end{equation*}
    Notacion:
    \begin{equation*}
        X \sim Ber(p)
    \end{equation*}
    Se lee, "la varaibale aleatoria $X$ tiene distribucion de Bernoulli con parametro $p$"
\end{definition}

\begin{definition}[Funcion Indicadora]
    Un caso particular, es la funcion indicadore, que es una forma "compacta" de escribir una funcion partida
    \begin{equation*}
        \mathbb{1}\{X \in A\} = 
        \left\{
           \begin{aligned}
                &0 &X \notin A\\
                &1 &X \in A
           \end{aligned}
        \right.
    \end{equation*}
\end{definition}

\textbf{IMPORTANTE} Para trabajar en la guia 2, el procedimiento es:
\begin{center}
    Identificar modelos $\rightarrow$ Tabla de Distribuciones
\end{center}
$V.A.D.$:
\begin{center}
    \begin{enumerate}
        \item Encuentro la $p_{X}(x)$
        \item Voy a la tabla y veo a quien pertenece.
    \end{enumerate}
\end{center}

\begin{example*}
    \begin{equation*}
        p_{X}(x) = \frac{\binom{7}{x}\binom{4}{5-x}}{\binom{11}{5}}, x = 1, 2, 3, 4, 5
    \end{equation*}
\end{example*}

\begin{definition}[Variables Aleatorias Continuas]
    Una variable aleatoria es continua si se cumplen las siguientes condiciones
    \begin{enumerate}
        \item Su conjunto de valores posibles se compone de todos los numeros que hay en un solo intervalo, o en una union exdcluyente de intervalos.
        \item Ningun valor posible de la variable aleatoria tiene probabilidad positiva, es decir $P(X = C) = 0$ $\forall{C \in \mathbb{R}}$
    \end{enumerate}
\end{definition}

\begin{definition}[Funcion de Densidad de Probabilidad]
    Se dice que $X$ es una $V.A.C.$ si existe una funcion $f_{X}: \mathbb{R} \rightarrow \mathbb{R}$, llamada \textbf{funcion de densidad de probabilidad}, que satisface las siguientes condiciones
    \begin{enumerate}
        \item $f_{X} \geq 0$, $\forall{x} \in \mathbb{R}$
        \item $\int_{-\infty}^{\infty}f_{X}(x)dx = 1$
        \item Para cualquier $a$ y $b$ tales que $-\infty < a < b < +\infty$ tenemos
        \begin{equation*}
            P(a < X < b) = \int_{a}^{b}f_{X}(x)dx
        \end{equation*}
    \end{enumerate}
    \begin{center}
        GRAFICO
    \end{center}
\end{definition}

\begin{defexample}
    La demanda de aceite pesado en cientos de litros durante una temporada tiene la siguiente funcion de densidad
    \begin{equation*}
        f_{X}(x) = \frac{4x + 1}{3} \cdot \mathbb{1}\{0<x<1\}
    \end{equation*}
    \begin{enumerate}
        \item Graficar $f_{X}(x)$ y verificar que sea una funcion de densidad.
        \item Hallar la funcion de distribucion de $X$.
        \item Calcular $P(\frac{1}{3} < X \leq \frac{2}{3})$ y $P(\frac{1}{3} < X \leq \frac{2}{3}|X < \frac{1}{2})$
    \end{enumerate}
    \begin{enumerate}
        \item 
        \textbf{Paso 1:} Defino una variable aleatoria.
        \begin{center}
            X: "Demanda de aceite en cientos de litros".
        \end{center}
        \textbf{Paso 2:} Grafico y verifico si es de densidad
        \begin{equation*}
            f_{X}(x) = 
            \left\{
                \begin{aligned}
                    &\frac{4x + 1}{3} &0<x<1\\
                    &0 &e.o.c.
                \end{aligned}
            \right.
        \end{equation*}
        \begin{enumerate}
            \item $f_{X} \geq 0$. SI
            \item $\int_{-\infty}^{\infty}f_{X}(x)dx = 1$. SI (calculas la integral o calcular el area de la figura bajo la curva, en este caso, el area del trapecio).
        \end{enumerate}
        \item 
        \begin{equation*}
            \begin{aligned}
                F_{X}(x) &= P(X \leq x)\\
                         &= \int_{-\infty}^{x}f_{X}(t)dt\\
                         &=
                         \left\{
                             \begin{aligned}
                                &\int_{-\infty}^{x}0dx = 0 &x< 0\\
                                &\int_{-\infty}^{0}0dx + \int_{0}^{x}\frac{4t + 1}{3}dt &0 \leq x < 1\\
                                &1 &x\geq 1
                             \end{aligned}
                         \right.\\
                         &= 
                         \left\{
                            \begin{aligned}
                               &0 &x< 0\\
                               &2x^{2} + x &0 \leq x < 1\\
                               &1 &x\geq 1
                            \end{aligned}
                         \right.\\
            \end{aligned}
        \end{equation*}
        Ahora graficamos la funcion de probabilidad para verificar si cumple con las condiciones.
    \end{enumerate}
\end{defexample}

\begin{note}
    Si $X$ es una $V.A.C.$, $F_{X}(x)$ es una funcion continua $\forall{x} \in \mathbb{R}$.
\end{note}

\begin{theorem}
    Sea $F_{X}(x)$ la funcion de distribucion de una $V.A.C.$ (admite derivada), luego
    \begin{equation*}
        f_{X}(x) = \frac{\partial}{\partial{x}} F_{X}(x)
    \end{equation*}
\end{theorem}

\begin{note}
    La funcion de densidad solo existe para $V.A.C.$
\end{note}

\subsubsection{Tipo de Variable Segun su Funcion de Distribucion}

\subsubsection{Atomo}
\begin{definition}[Atomo]
    Diremos que $a \in \mathbb{R}$ es un atomo de $F_{X}(x)$ si su peso es positivo, es decor $P(X = a) > 0$.
    El conjunto de todos los atomos de $F_{X}(x)$ coincide con todos los puntos de discontinuidad de $F_{X}(x)$. Podemos definir:
    \begin{enumerate}
        \item La $V.A.$ $X$ sera discreta si la suma de las probabilidades de todos los atomos es 1.
        \item La $V.A.$ $X$ sera continua, si $F_{X}(x)$ es continua (el conjunto de atomos es vacio).
        \item La $V.A.$ $X$ sera \textbf{MIXTA} si no es ni continua, ni discreta.
    \end{enumerate}
\end{definition}

\subsubsection{Soporte}
\begin{definition}[Soporte]
    \begin{equation*}
        S_{X} = \{x \in \mathbb{R}: F_{X}(x) - F_{X}(\overline{x}) \neq 0 \text{ o } \frac{\partial}{\partial{x}}F_{X}(x) \neq 0 \}
    \end{equation*}
\end{definition}

\begin{example*}
    Sea $X$ una $V.A.$ con funcion de distribucion
    \begin{equation*}
        \begin{aligned}
            F_{X}(x) &= P(X \leq x)\\
                     &= \frac{x}{4} \cdot \mathbb{1}\{0 \leq x < 1\} + \frac{1}{3} \cdot \mathbb{1}\{1 \leq x < 2\} + \frac{x}{6} \cdot \mathbb{1}\{x \geq 4\}
        \end{aligned}
    \end{equation*}
    \begin{enumerate}
        \item Hallar
        \begin{equation*}
            P(1 < X \leq 4), P(1 \leq X \leq 4), P(1 \leq X < 4), P(1 < X < 4) 
        \end{equation*}
        \item Sean $A = [1, 3.5]$, $B = (0.5, 3)$ calcular
        \begin{equation*}
            P(X \in A), P(X \in B|X \in A)
        \end{equation*}
    \end{enumerate}
\end{example*}