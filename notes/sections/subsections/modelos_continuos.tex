\documentclass[../../main.tex]{subfiles}

Puntos al azar en un continuo.

El experimento consiste en elegir un numero al azar en el intervalo [0, 1]

\begin{enumerate}
    \item Hallar la probabilidad de que los primeros 3 dígitos sean 3, 1, 4 (es decir 0.314...)
    \item Hallar la probabilidad de que el 0 no esté entre los priemros 4 dígitos.
\end{enumerate}

Lo que necesitamos es alguna manera de calcular la probabilidad para estos experimentos

\begin{proposition}
    \begin{equation*}
        P(x \in [a; b]) = b - a
    \end{equation*}
\end{proposition}

\begin{proof} Veo si cumple con los axiomas
    \begin{enumerate}
        \item $0 \leq P(A) \leq 1$, $\forall A \in \mathcal{A}$.
        \item $P(\Omega) = 1$.
        \item $A \cap B = \O \rightarrow P(A \cap B) = P(A) + P(B)$ 
    \end{enumerate}
\end{proof}

Entonces para mi caso

\begin{enumerate}
    \item 
        \begin{equation*}
            P(x \in [0.314; 0.315)) = 0.315 - 0.314 = 0
        \end{equation*}
    \item No quiero al 0 en los primeros 4.
    \textit{O: "No aparece el 0 en los primeros 4 dígitos"}
    Lo que voy a ir haciendo es ir quitando los intervalos en donde hay un 0 en los primeros 4 dígitos, que valores de todo el segmento tienen como primer dígito un 0, eso corresponde al intervalo [0, 0.1). Haciendo lo mismo para los que tienen 0 en el segundo dígito, entonces [0.1, 0.11), [0.2, 0.21), ..., [0.9, 0.91).
    \begin{equation*}
        P(O) = 1 - P(x \in [0; 0.1))  P(x \in [0.1; 0.11)) \cdot 9 - \dots
    \end{equation*}
    \begin{equation*}
        \begin{aligned}
            P(O) &= 1 - 0.1 - 9 \cdot 0.01 - 9^{2} \cdot 0.001 - 9^{3} \cdot 0.0001\\
            &= 1 - \frac{1}{10} - - \frac{9}{10} \cdot \frac{1}{10} - \left(\frac{9}{10}\right)^{2} \cdot \frac{1}{10} - \left(\frac{9}{10}\right)^{3} \cdot \frac{1}{10}\\
            &= 1 - \sum_{i=0}^{3} \frac{1}{10} \cdot \left(\frac{9}{10}\right)^{i}
            &=
        \end{aligned}
    \end{equation*}
\end{enumerate}