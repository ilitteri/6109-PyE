\documentclass[../../main.tex]{subfiles}

\begin{definition}[Eventos Independientes]
    Sea $(\Omega, \mathcal{A}, P)$ un espacio de probabilidad, $A, B \in \mathcal{A}$ los eventos independientes si y sólo si
    \begin{equation*}
        P(A \cap B) = P(A) \cdot P(B)
    \end{equation*}
\end{definition}

\begin{properties} Eventos independientes
    \begin{enumerate}
        \item Si $A$ y $B$ son independientes, también lo son $A$ y $\overline{B}$, $\overline{A}$ y $B$, $\overline{A}$ y $\overline{B}$. \textit{demostrar!!!}
        \item $A_{1}, \dots, A_{n}$ son independientes si y sólo si para cada sucesión de $k$ conjuntos $2 \leq k \leq n$, la probabilidad de la intersección de los $k$ sucesos coincide con el producto de las probabilidades.
    \end{enumerate}
\end{properties}

\begin{defexample}
    Supongo que $A$, $B$, y $C$.
    Para que sean independientes tiene que ocurrir que:
    \begin{equation*}
        \left.
            \begin{aligned}
                P(A \cap B) &= P(A) \cdot P(B)\\
                P(A \cap C) &= P(A) \cdot P(C)\\
                P(B \cap C) &= P(B) \cdot P(C)\\
            \end{aligned}
        \right\}
        \quad k=2
    \end{equation*}
    \begin{equation*}
        P(A \cap B \cap C) = P(A) \cdot P(B)
        \quad k=3
    \end{equation*}
\end{defexample}

\begin{observation}
    Si $P(B) > 0$, cuando $A$ y $B$ son independientes ocurre que 
    \[P(A|B)\eqdownarrowx[1.5]{P(B) > 0}\frac{P(A \cap B)}{P(B)} = \frac{P(A) \cdot P(B)}{P(B)} = P(A)\]
\end{observation}

\begin{defexample}
    Se elije al azar una permutación de las letras A, T, C, G. Mostrar que los eventos \textit{"A pertence a T"} y \textit{"C precede a G"} son independientes.

    Primero elijo mi experimento aleatorio
    \begin{center}
        E.A: Elijo una permutación de las letras ATCG,
    \end{center}
    como las letras son todas distintas entonces
    \begin{equation*}
        \#CP = 4! = 24
    \end{equation*}
    Ahora defino efentos
    \begin{center}
        \begin{align*}
            AT:& "A precede a T"\\
            CG:& "C precede a T"
        \end{align*}
    \end{center}
    Mis eventos van a ser independientes tengo que calcular tres probabilidades y ver si se cumple. Como mi espacio es equiprobable, puedo usar Laplace.
    \begin{center}
        \begin{align*}
            ATCG\\
            ACTG\\
            ACGT\\
            CATG\\
            CATG\\
            CGAT
        \end{align*}
    \end{center}
    veo que tengo 6 casos, e imagino que si ponía $GC$ tendría otros 6 casos, y si ponía $TA$ tendría otros 12 casos. De esta forma veo los 24 casos.
    \begin{equation*}
        \begin{aligned}
            &\#AT = 6 + 6 = 12\\
            &\#CG = 12\\
            &\#(AT \cap CG) = 6 
        \end{aligned}
    \end{equation*}
    Sólo resta calcular las probabilidades
    \begin{equation*}
        \begin{aligned}
            &P(AT \cap CG) = \frac{6}{24} = \frac{1}{4}\\
            &P(AT) \cdot P(CG) = \frac{12}{24} \cdot \frac{12}{24} = \frac{1}{4}\\
        \end{aligned}
    \end{equation*}
\end{defexample}