\documentclass[../../main.tex]{subfiles}

\begin{definition}[Álgebra de eventos]
    Dado $\Omega$, sea $\mathcal{A}$ una familia de subconjuntos de $\Omega$, diremos que $\mathcal{A}$ es un álgebra de eventos si:
    \begin{enumerate}
        \item $\Omega \in \mathcal{A}$ .
        \item Si $B \in \mathcal{A} \Rightarrow \overline{B} \in \mathcal{A}$.
        \item Si $B, C \in \mathcal{A} \Rightarrow B \cup C \in \mathcal{A}$.
        \item ($\sigma$ Álgebra) Si $(A_{n})_{n \geq 1}$ es una sucesión en $\mathcal{A}$ entonces:
            \begin{equation*}
                \bigcup_{i = 1}^{\infty} A_{i} \in \mathcal{A}
            \end{equation*}
    \end{enumerate} 
\end{definition}

\begin{properties} Álgebra de eventos
    \begin{enumerate}
        \item $\O \in \mathcal{A}$
        \item Si $A_{1}, \dots, A_{n} \in \mathcal{A} \Rightarrow \bigcup_{i=1}^{n}A_{i} \in \mathcal{A}$
        \item Si $A_{1}, \dots, A_{n} \in \mathcal{A} \Rightarrow \bigcap_{i=1}^{n}A_{i} \in \mathcal{A}$
    \end{enumerate}
\end{properties}

\begin{defexample}
    Sea $\Omega = \{1, 2, 3, 4, 5, 6\}$. Hallar la menor álgebra de subconjuntos tal que el subconjunto $\{2, 4, 6\}$ pertenezca a ella.\\
    Solución:\\
    Por axioma 2 debe estar su complemento ($\{1, 3, 5\}$).\\
    Por axioma 3 debe estar la unión ($\{1, 2, 3, 4, 5, 6\}$).\\
    Por axioma 2 debe estar su complemento ($\{\O\}$).\\
    El axioma 1 se cumplió por accidente.
    \begin{equation*}
        \mathcal{A} = \{\{2, 4, 6\}, \{1, 3, 5\}, \{1, 2, 3, 4, 5, 6\}, \{\O\}\}
    \end{equation*}
    De eta forma encontramos el menor álgebra de subconjuntos de $\Omega$ con la condición pedida.
\end{defexample}

\begin{definition}[Calcular probabilidad en cualquier espacio]
    Una probabilidad (o medida de probabilidad) e suna funcion $P: \mathcal{A} \rightarrow [0, 1]$ que a cada evento $A$ le hace corresponder un número real $P(A)$ con las siguientes propiedades:
    \begin{enumerate}
        \item $0 \leq P(A) \leq 1, \forall A \in \mathcal{A}$ (la probabilida es un número entre 0 y 1).
        \item $P(\Omega) = 1$.
        \item Si $A \cap B = \O \Rightarrow P(A \cup B) = P(A) + P(B)$ (eventos mutuamente excluyentes o disjuntos: no pueden ocurrir al mismo tiempo).
        \item (Axioma de continuidad) Para cada sucesión decreciente de eventos $A_{1} \subseteq A_{2} \subseteq A_{3} \subseteq \dots$ tal que $\bigcap_{i=1}^{infty}A_{i} = \O$ vale que $\lim_{x\to{infty}}P(A_{n}) = 0$. 
    \end{enumerate}
\end{definition}

\begin{properties} :
    \begin{enumerate}
        \item Si $\overline{A}$ es el evento complementario de $A$, entonces $P(\overline{A}) = 1 - P(A)$.
            \begin{proof}[Demostración] (usando los axiomas)\\
                \begin{center}
                    $\begin{aligned}
                        \Omega =& A \cup \overline{A}\\
                        &A \cap \overline{A} = \O\\
                        \text{ax.2) } P(\Omega) =& P(A \cup \overline{A}) = 1\\
                        \text{ax.3) } P(A \cup \overline{A}) =& P(A) + P(\overline{A}) = 1\\
                        &\therefore P(\overline{A}) = 1 - P(A)
                    \end{aligned}$
                \end{center}
            \end{proof}
        \item Sean $A$ y $B$ eventos pertenecientes a $\Omega$, entonces:
            \begin{equation*}
                P(A \cup B) = P(A) + P(B) - P(A \cap B)
            \end{equation*}
            \begin{proof}[Demostración] usando los axiomas:
                \begin{center}
                    $\begin{aligned}
                        Completar\\
                    \end{aligned}$
                \end{center}
            \end{proof}
        \item Si $A_{1}, A_{2}, \dots, A_{n}$ es una sucesión de elementos de $A$, mutuamente excluyentes $2$ a $2$, entonces
            \begin{equation*}
                P\left(\bigcup_{i=1}^{n}A_{i}\right) = \sum_{i=1}^{n}P(A_{i})
            \end{equation*}
    \end{enumerate}
\end{properties}

\begin{note}[¿Que significa para nosotros el álgebra de eventos?]
    Cuando defino un experimento aleatorio puedo definir $\Omega$ pero no necesariamente puedo conocer todas las probabilidades de todos los elementos elementales, puede que me falte información, entonces el conjunto del álgebra de eventos me dice a que eventos les puedo calcular la probabilidad.
\end{note}

\begin{example*}
    En argentina, el 80\% de los programadores usa Java, C, o ambos. El 50\% usa Java y el 4'\% usa C. ¿ Cuál es la probabilidad de que al elegir un programador al azar use:
    \begin{enumerate}[label=\alph*)]
        \item Java y C:
        \item Sólo Java.
        \item Solo C.
        \item Ninguno de los dos lenguajes.
    \end{enumerate}

    Solución:

    \begin{itemize}
        \item Elijo el experimento aleatorio (EA):\\ "Elijo un programador al azar y me pregunto que lenguaje usa".
        \item Defino eventos:\\
        $J$: "El programador elegido al azar usa Java".\\
        $C$: "El programador elegido al azar usa C".\\
        \item Defino quién es $\Omega$ (en ese caso lo defino con un diagrama, en específico un diagrama de Venn):
            \subfile{../../diagrams/venn_diagram.tex}
        \item Interpreto los datos:
            % \begin{center}
            %     $\begin{aligned}
            %         P(J \cup C) = 0.8\\
            %         P(J) = 0.5\\
            %         P(C) = 0.4\\
            %         \therefore P(J \cup C) = P(J) + P(C) - P(J \cap C)\\
            %         \Rightarrow P(J \cap C) = P(J) + P(C) - P(J \cup C) = 0.1
            %     \end{aligned}$
            % \end{center}
            \begin{equation*}
                P(J \cup C) = 0.8
            \end{equation*}
            \begin{equation*}
                P(J) = 0.5
            \end{equation*}
            \begin{equation*}
                P(C) = 0.4
            \end{equation*}
            \begin{equation*}
                \therefore P(J \cup C) = P(J) + P(C) - P(J \cap C)
            \end{equation*}
            \begin{equation*}
                \Rightarrow P(J \cap C) = P(J) + P(C) - P(J \cup C) = 0.1
            \end{equation*}
        \item Resuelvo
            \begin{enumerate}[label=\alph*)]
                \item $P(J \cap C) = 0.1$
                \item $P(J \cap \overline{C}) = P(J - C \cap J) = P(J) - P(C \cap J) = 0.4$
                \item $P(C \cap \overline{J}) = P(C - J \cap C) = P(C) - P(J \cap C) = 0.3$
                \item "Ninguno" es el complemento de "alguno" y "alguno" $= (J \cup C)$ entonces:
                    \begin{equation*}
                        P(\overline{J \cup C}) = P(\overline{C} \cap \overline{J}) = 1 - P(C \cup J) = 0.2
                    \end{equation*}
            \end{enumerate}
    \end{itemize}
\end{example*}

\begin{definition}[Espacio de probabilidad]
    Es una terna $(\Omega, \mathcal{A}, P)$, donde $\Omega$ es un conjunto no vacío, $\mathcal{A}$ es un álgebra sobre $\Omega$ y $P$ es una medida de probabilidad.
\end{definition}

\begin{theorem}
    Sea $(A_{n})_{n\geq1}$ una suscesión de eventos tales que $A_{n} \subset A_{n+1}$ $\forall n$ y $A = \bigcup_{i = 1}^{\infty} A_{i}$, luego
        \begin{equation*}
            P(A) = \lim_{n\rightarrow \infty} P(A_{n})
        \end{equation*}
        Sea $(A_{n})_{n\geq1}$ una suscesión de eventos tales que $A_{n+1} \subset A_{n}$ $\forall n$ y $A = \bigcap_{i = 1}^{\infty} A_{i}$, luego
            \begin{equation*}
                P(A) = \lim_{n\rightarrow \infty} P(A_{n})
            \end{equation*}
\end{theorem}

\begin{theorem}[$\sigma$-aditividad]
    Sea $A = \bigcup_{i = 1}^{\infty} A_{i} \in \mathcal{A}$ con los eventos $A_{i}$ mutuamente excluyentes 2 a 2 (cualquier par), entonces
        \begin{equation*}
            P(A) = P\left(\bigcup_{i = 1}^{\infty} A_{i}\right) = \sum_{i=1}^{\infty} P(A)
        \end{equation*}
\end{theorem}