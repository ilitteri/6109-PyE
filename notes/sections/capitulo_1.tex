\documentclass[../main.tex]{subfiles}

\begin{document}

El término probabilidad se refiere al término de azar, y la incertidumbre en cualquier situación en la que varios resultados pueden ocurrir.

\begin{definition}[Experimentos aleatorios]
    Acciones o procesos en los cuales conocemos todos los resultados posibles pero no sabemos con certeza cuál va a ocurrir.
\end{definition}

Si conocemos todos los resultados posibles entonces podemos anotarlos, entonces definimos *espacio muestral*:

\begin{definition}[Espacio muestral ($\Omega$)]
    Es el conjunto de todos los resultados posibles del experimento aleatorio. Sus elementos, $\omega$, se llaman \textbf{elementos elementales}.
\end{definition}

\begin{defexamples} Casos que conozco todas los posibles resultados pero no el resultado final:
    \begin{enumerate}
        \item Tiro una moneda y observo la cara superior.\\
        Espacio muestral:
        \begin{equation*}
            \Omega_{1} = \{"cara", "ceca"\}
        \end{equation*}
        \item Tiro una moneda 2 veces y observo que sale.\\
        Espacio muestral:
        \begin{equation*}
            \Omega_{2} = \{("cara", "ceca"), ("ceca", "cara"), ("cara", "cara"), ("ceca", "ceca")\}
        \end{equation*}
        \item Tiro un dado y observo el resultado.\\
        Espacio muestral:
        \begin{equation*}
            \Omega_{3} = \{1, 2, 3, 4, 5, 6\}
        \end{equation*}
        \item Registro la cantidad de personas que entran a un banco entre las 11 y las 12hs.\\
        Espacio muestral:
        \begin{equation*}
            \Omega_{4} = \{0, 1, 2, 3, \dots\} = \mathbb{N}_{0}
        \end{equation*}
        \item Registro el tiempo entre la llegada de autos a un peaje.\\
        Espacio muestral:
        \begin{equation*}
            \Omega_{5} = \{t: t \in \mathbb{R}, t \geq 0\}
        \end{equation*}
    \end{enumerate}
\end{defexamples}

En el estudio de la probabilidad nos interesa no solo los resultados individuales de los espacios muestrales sino que nos interesan varias recopilaciones de resultados. Por eso definimos \textit{evento} o \textit{suceso}:

\begin{definition}[Evento o Suceso]
    Es cualquier conjunto de resultados en el espacio muestral. Los resultados pueden mostrar un conjunto finito o infinito con cualquier cardinalidad.
\end{definition}

\begin{example*} Refiriendonos a (1) podemos definir un evento "A" como:
    \begin{enumerate}
        \item A. "El valor observado es par". (está formado por 3 eventos elementales)\
        Asi creamos un subconjunto que corresponde a los elementos de $\Omega_{1}$ que cumple con el evento "A".
    \end{enumerate}
\end{example*}

\begin{definition}[Espacio equiprobable]
    Un espacio muestrable es equiprobable cuando todos sus elementos tienen la misma probabilidad de ser elegidos.
\end{definition}

\begin{definition}[Frecuencia absoluta]
    Para un evento en particular, la frecuencia absoluta es la cantidad de veces que sucede ese evento. La cantidad de veces que sucede el evento A (o \#A), se nota:
    \begin{equation*}
        \eta_{A}
    \end{equation*}
\end{definition}

\begin{definition}[Frecuencia relativa]
    Para un evento en particular, se define como la relación entre la cantidad de veces que ocurre el evento A y el número total de ensayos.
    \begin{equation*}
        f_{a} = \frac{\eta_{A}}{\eta}
    \end{equation*}
\end{definition}

Ahora estamos en condiciones para definir \textit{probabilidad}.

\begin{definition}[Probabilidad]
    Probabilidad de un evento A, es un número positivo (o nulo) que se le asigna a cada suceso o evento del espacio muestral.
\end{definition}

\begin{definition}[Regla de Laplace]
    La probabilidad de que ocurra un sucedo A se calcula como la cantidad de casos en los que ocurre ese suceso dividido los casos posbiles de ese experimento siempre y cuando los todos los elementos del espacio muestral sean equiprobables:
    \begin{equation*}
        P(A) = \frac{\#\text{casos favorables de A}}{\#\text{casos posibles del experimento}}
    \end{equation*}
\end{definition}

\begin{defexamples} Regla de Laplace
    \begin{enumerate}
        \item Arrojo un dado equilibrado ¿cuál es la probabilidad de que observe el número 2? ¿cuál es la probabilidad de que observe un número par?\\
        Solución:
        \begin{itemize}
            \item Experimento aleatorio: arrojo un dado y observo el resultado.
            \item Espacio muestral:
                \begin{equation*}
                    \Omega = \{1, 2, 3, 4, 5, 6\}
                \end{equation*}
            \item Evento A. "Se observa el número 2".
            \item Definición de Laplace: ¿es mi espacio equiprobable? el dado es equilibrado, por lo tanto mi espacio es equiprobable. Entonces puedo usar la definición:
                \begin{equation*}
                    P(A) = \frac{\mid A\mid}{\mid\Omega\mid} \therefore \mid A\mid = 1 \wedge \mid\Omega\mid = 6 \Rightarrow P(A) = \frac{1}{6}
                \end{equation*}
            \item Evento B. "Se observa un número par".
            \item 
                \begin{equation*}
                    P(B) = \frac{\mid B\mid}{\mid\Omega\mid} = \frac{3}{6} = \frac{1}{2}.
                \end{equation*}
        \end{itemize}
        \item  Un dado equilibrado se arroja 2 veces. Hallar la probabilidad de que:
            \begin{enumerate}
                \item  Los dos resultados sean iguales.
                \item Los dos resultados sean distintos y su suma no supere 9.
                \item La suma de los resultados sea 10.
                \item El primr resultado sea 4 y el segundo resultado sea impar.
            \end{enumerate}
            Solución:
            \begin{itemize}
                \item Experimento aleatorio: arrojo un dado 2 veces y observo el resultado.
                \item Evento $D_{i}$: "Valor observado en el tiro $i$" $i = 1, 2$
                \item Como los resultados son muchos para escribir el conjunto entero, escribo una tabla:
                    \begin{equation*}
                        \Omega = \{(a, b) : a, b = \{1, 2, 3, 4, 5, 6\}\} \therefore \mid\Omega\mid = 36
                    \end{equation*}
                    \begin{table}[H]
                        \begin{center}
                            \begin{tabular}{c|c|c|c|c|c|c}
                                $D_{2}/D_{1}$ & $1$ & $2$ & $3$ & $4$ & $5$ & $6$\\
                                \hline
                                $1$ & \cellcolor{orange} & \cellcolor{cyan} & \cellcolor{cyan} & \cellcolor{cyan}{$\cdot$} & \cellcolor{cyan} & \cellcolor{cyan}\\
                                \hline
                                $2$ & \cellcolor{cyan} & \cellcolor{orange} & \cellcolor{cyan} & \cellcolor{cyan} & \cellcolor{cyan} & \cellcolor{cyan}\\
                                \hline
                                $3$ & \cellcolor{cyan} & \cellcolor{cyan} & \cellcolor{orange} & \cellcolor{cyan}{$\cdot$} & \cellcolor{cyan} & \cellcolor{cyan}\\
                                \hline
                                $4$ & \cellcolor{cyan} & \cellcolor{cyan} & \cellcolor{cyan} & \cellcolor{orange} & \cellcolor{cyan} & \cellcolor{pink}\\
                                \hline
                                $5$ & \cellcolor{cyan} & \cellcolor{cyan} & \cellcolor{cyan} & \cellcolor{cyan}{$\cdot$} & \cellcolor{orange} & \cellcolor{pink}\\
                                \hline
                                $6$ & \cellcolor{cyan} & \cellcolor{cyan} & \cellcolor{cyan} & \cellcolor{pink} & \cellcolor{pink} & \cellcolor{orange}\\
                            \end{tabular}
                        \end{center}
                    \end{table}
                \end{itemize}
            \begin{enumerate}
                \item \colorbox{orange}{A}: "Los dos resultados son iguales" $\therefore \mid\colorbox{orange}{A}\mid = 6$ 
                \begin{equation*}
                    P(A) = \frac{\mid A\mid}{\mid\Omega\mid} = \frac{6}{36} = \frac{1}{6}
                \end{equation*}
                \item \colorbox{cyan}{B}: "Los resultados son distintos y la suma no supera 9" $\therefore \mid\colorbox{cyan}{B}\mid = 26$
                    \begin{equation*}
                        P(B) = \frac{\mid B\mid}{\mid\Omega\mid} = \frac{26}{36} = \frac{13}{26}
                    \end{equation*}
                \item  \colorbox{pink}{C}: "La suma de los resultados sea 10s" $\therefore \mid\colorbox{pink}{C}\mid = 4$
                    \begin{equation*}
                        P(C) = \frac{\mid C\mid}{\mid\Omega\mid} = \frac{4}{36} = \frac{1}{9}
                    \end{equation*}
                \item  \colorbox{cyan}{$\cdot D$}: "El primer resultado sea 4 y el segundo resultado sea impar". $\therefore \mid\colorbox{cyan}{$\cdot D$}\mid = 4$
                \begin{equation*}
                    P(D) = \frac{\mid D\mid}{\mid\Omega\mid} = \frac{3}{36} = \frac{1}{12}
                \end{equation*}
            \end{enumerate}

    \end{enumerate}
\end{defexamples}

\subsection{Preliminares}

\begin{definition}[Álgebra de eventos]
    Dado $\Omega$, sea $\mathcal{A}$ una familia de subconjuntos de $\Omega$, diremos que $\mathcal{A}$ es un álgebra de eventos si:
    \begin{enumerate}
        \item $\Omega \in \mathcal{A}$ .
        \item Si $B \in \mathcal{A} \Rightarrow \overline{B} \in \mathcal{A}$.
        \item Si $B, C \in \mathcal{A} \Rightarrow B \cup C \in \mathcal{A}$.
        \item ($\sigma$ Álgebra) Si $(A_{n})_{n \geq 1}$ es una sucesión en $\mathcal{A}$ entonces:
            \begin{equation*}
                \bigcup_{i = 1}^{\infty} A_{i} \in \mathcal{A}
            \end{equation*}
    \end{enumerate} 
\end{definition}

\begin{properties} Álgebra de eventos
    \begin{enumerate}
        \item $\O \in \mathcal{A}$
        \item Si $A_{1}, \dots, A_{n} \in \mathcal{A} \Rightarrow \bigcup_{i=1}^{n}A_{i} \in \mathcal{A}$
        \item Si $A_{1}, \dots, A_{n} \in \mathcal{A} \Rightarrow \bigcap_{i=1}^{n}A_{i} \in \mathcal{A}$
    \end{enumerate}
\end{properties}

\begin{defexample}
    Sea $\Omega = \{1, 2, 3, 4, 5, 6\}$. Hallar la menor álgebra de subconjuntos tal que el subconjunto $\{2, 4, 6\}$ pertenezca a ella.\\
    Solución:\\
    Por axioma 2 debe estar su complemento ($\{1, 3, 5\}$).\\
    Por axioma 3 debe estar la unión ($\{1, 2, 3, 4, 5, 6\}$).\\
    Por axioma 2 debe estar su complemento ($\{\O\}$).\\
    El axioma 1 se cumplió por accidente.
    \begin{equation*}
        \mathcal{A} = \{\{2, 4, 6\}, \{1, 3, 5\}, \{1, 2, 3, 4, 5, 6\}, \{\O\}\}
    \end{equation*}
    De eta forma encontramos el menor álgebra de subconjuntos de $\Omega$ con la condición pedida.
\end{defexample}

\begin{definition}[Calcular probabilidad en cualquier espacio]
    Una probabilidad (o medida de probabilidad) e suna funcion $P: \mathcal{A} \rightarrow [0, 1]$ que a cada evento $A$ le hace corresponder un número real $P(A)$ con las siguientes propiedades:
    \begin{enumerate}
        \item $0 \leq P(A) \leq 1, \forall A \in \mathcal{A}$ (la probabilida es un número entre 0 y 1).
        \item $P(\Omega) = 1$.
        \item Si $A \cap B = \O \Rightarrow P(A \cup B) = P(A) + P(B)$ (eventos mutuamente excluyentes o disjuntos: no pueden ocurrir al mismo tiempo).
        \item (Axioma de continuidad) Para cada sucesión decreciente de eventos $A_{1} \subseteq A_{2} \subseteq A_{3} \subseteq \dots$ tal que $\bigcap_{i=1}^{infty}A_{i} = \O$ vale que $\lim_{x\to{infty}}P(A_{n}) = 0$. 
    \end{enumerate}
\end{definition}

\begin{properties} :
    \begin{enumerate}
        \item Si $\overline{A}$ es el evento complementario de $A$, entonces $P(\overline{A}) = 1 - P(A)$.
            \begin{proof}[Demostración] (usando los axiomas)\\
                \begin{center}
                    $\begin{aligned}
                        \Omega =& A \cup \overline{A}\\
                        &A \cap \overline{A} = \O\\
                        \text{ax.2) } P(\Omega) =& P(A \cup \overline{A}) = 1\\
                        \text{ax.3) } P(A \cup \overline{A}) =& P(A) + P(\overline{A}) = 1\\
                        &\therefore P(\overline{A}) = 1 - P(A)
                    \end{aligned}$
                \end{center}
            \end{proof}
        \item Sean $A$ y $B$ eventos pertenecientes a $\Omega$, entonces:
            \begin{equation*}
                P(A \cup B) = P(A) + P(B) - P(A \cap B)
            \end{equation*}
            \begin{proof}[Demostración] usando los axiomas:
                \begin{center}
                    $\begin{aligned}
                        Completar\\
                    \end{aligned}$
                \end{center}
            \end{proof}
        \item Si $A_{1}, A_{2}, \dots, A_{n}$ es una sucesión de elementos de $A$, mutuamente excluyentes $2$ a $2$, entonces
            \begin{equation*}
                P\left(\bigcup_{i=1}^{n}A_{i}\right) = \sum_{i=1}^{n}P(A_{i})
            \end{equation*}
    \end{enumerate}
\end{properties}

\begin{note}[¿Que significa para nosotros el álgebra de eventos?]
    Cuando defino un experimento aleatorio puedo definir $\Omega$ pero no necesariamente puedo conocer todas las probabilidades de todos los elementos elementales, puede que me falte información, entonces el conjunto del álgebra de eventos me dice a que eventos les puedo calcular la probabilidad.
\end{note}

\begin{example*}
    En argentina, el 80\% de los programadores usa Java, C, o ambos. El 50\% usa Java y el 4'\% usa C. ¿ Cuál es la probabilidad de que al elegir un programador al azar use:
    \begin{enumerate}[label=\alph*)]
        \item Java y C:
        \item Sólo Java.
        \item Solo C.
        \item Ninguno de los dos lenguajes.
    \end{enumerate}

    Solución:

    \begin{itemize}
        \item Elijo el experimento aleatorio (EA):\\ "Elijo un programador al azar y me pregunto que lenguaje usa".
        \item Defino eventos:\\
        $J$: "El programador elegido al azar usa Java".\\
        $C$: "El programador elegido al azar usa C".\\
        \item Defino quién es $\Omega$ (en ese caso lo defino con un diagrama, en específico un diagrama de Venn):
            \subfile{../diagrams/venn_diagram.tex}
        \item Interpreto los datos:
            % \begin{center}
            %     $\begin{aligned}
            %         P(J \cup C) = 0.8\\
            %         P(J) = 0.5\\
            %         P(C) = 0.4\\
            %         \therefore P(J \cup C) = P(J) + P(C) - P(J \cap C)\\
            %         \Rightarrow P(J \cap C) = P(J) + P(C) - P(J \cup C) = 0.1
            %     \end{aligned}$
            % \end{center}
            \begin{equation*}
                P(J \cup C) = 0.8
            \end{equation*}
            \begin{equation*}
                P(J) = 0.5
            \end{equation*}
            \begin{equation*}
                P(C) = 0.4
            \end{equation*}
            \begin{equation*}
                \therefore P(J \cup C) = P(J) + P(C) - P(J \cap C)
            \end{equation*}
            \begin{equation*}
                \Rightarrow P(J \cap C) = P(J) + P(C) - P(J \cup C) = 0.1
            \end{equation*}
        \item Resuelvo
            \begin{enumerate}[label=\alph*)]
                \item $P(J \cap C) = 0.1$
                \item $P(J \cap \overline{C}) = P(J - C \cap J) = P(J) - P(C \cap J) = 0.4$
                \item $P(C \cap \overline{J}) = P(C - J \cap C) = P(C) - P(J \cap C) = 0.3$
                \item "Ninguno" es el complemento de "alguno" y "alguno" $= (J \cup C)$ entonces:
                    \begin{equation*}
                        P(\overline{J \cup C}) = P(\overline{C} \cap \overline{J}) = 1 - P(C \cup J) = 0.2
                    \end{equation*}
            \end{enumerate}
    \end{itemize}
\end{example*}

\begin{definition}[Espacio de probabilidad]
    Es una terna $(\Omega, \mathcal{A}, P)$, donde $\Omega$ es un conjunto no vacío, $\mathcal{A}$ es un álgebra sobre $\Omega$ y $P$ es una medida de probabilidad.
\end{definition}

\subsection{Técnicas de Conteo}

\begin{definition}[\#CP]
    Cantidad de casos posibles de un experimento.
\end{definition}

Si tiramos un dado vemos que la cantidad de resultados posibles es $6$; si tiramos el dado $2$ veces, la cantidad de resultados posibles es $36$; si tiramos el dado $3$ veces, tengo $6$ opciones para cada tiro:

\begin{equation*}
    \underline{6}\text{ }\underline{6}\text{ }\underline{6}
\end{equation*}

\begin{definition}[Regla del Producto]
    Dados dos conjuntos $A$ y $B$ con $\eta_{A}$ y $\eta_{B}$ elementos cada uno respectivamente, la cantidad de todos los pares ordenados que pueden formarse con un elemento de $A$ y uno de $B$ se calcula como $\eta_{A} \cdot \eta_{B}$
\end{definition}

\begin{defexamples} Regla del producto
    \begin{enumerate}
        \item Tiro un dado 2 veces y por regla del producto:
            \begin{equation*}
                \#CP = \underline{6}\cdot\underline{6} = 36
            \end{equation*}
        \item  Tiro un dado 3 veces y por regla del producto:
            \begin{equation*}
                \#CP = \underline{6}\cdot\underline{6}\cdot\underline{6} = 6^{3}
            \end{equation*}
        \item Tiro un dado 4 veces y por regla del producto:
            \begin{equation*}
                \#CP = \underline{6}\cdot\underline{6}\cdot\underline{6}\cdot\underline{6} = 6^{4}
            \end{equation*}
        \item En 4 tiros, ¿de cuántas formas posibles puede no aparecer ningún 6?\\
        Tengo 5 opciones en donde no sale un 6, y como son 4 tiros, entonces tengo 5 opciones en donde no sale un 6, 4 veces. Entonces:
            \begin{equation*}
                \#CP = \underline{5}\cdot\underline{5}\cdot\underline{5}\cdot\underline{5} = 5^{4}
            \end{equation*}
        \item ¿Cuántas patentes de 3 letras y 3 números distintas puedo formar?\\
        Tengo 3 letras (una raya por letra) y 3 números (una raya por número). Tengo 26 letras posibles para la primer posición, para la segunda y tercera igual (porque se pueden repetir). Con los números, tengo 10 opciones para el primero, el segundo y el tercero (10 dígitos de 0 a 9). Por regla del producto:
            \begin{equation*}
                \#CP = \underline{26}\cdot\underline{26}\cdot\underline{26}\cdot\underline{10}\cdot\underline{10}\cdot\underline{10} = 26^{3}\cdot10^{3}
            \end{equation*}
    \end{enumerate}
\end{defexamples}

Supongamos que tengo 5 libros y quiero ordenarlos en una biblioteca, que sólo tiene 5 lugares. Quiero contar todas las formas posibles de ordenarlos. Entonces supongo un caso análogo a los anteriores.\\
Para el primer lugar tengo 5 posibles opciones para colocar uno de mis libros, para el segundo 4 opciones, para el tecero 3 opciones, para el cuarto 2 opciones y para el último lugar ya solo me queda una posible opción. Entonces:

\begin{equation*}
    \#CP = \underline{5}\cdot\underline{4}\cdot\underline{3}\cdot\underline{2}\cdot\underline{1} = 120 = 5!
\end{equation*}

Lo que hicimos fué pensar en la cantidad de libros que teníamos y ver la cantidad de \textit{ordenamientos} posibles.

\begin{definition}[Permutaciones]
    La cantidad de formas distintas en las que puedo ordenar n elementos es n!
\end{definition}

\begin{defexamples} Permutaciones
    \begin{enumerate}
        \item ¿De cuántas formas distintas puedo fotografiar a 7 personas en hilera?
            \begin{equation*}
                \#CP = \underline{7}\cdot\underline{6}\cdot\underline{5}\cdot\underline{4}\cdot\underline{3}\cdot\underline{2}\cdot\underline{1} = 7!
            \end{equation*}
        \item ¿Que pasaría si en una biblioteca tengo espacio para 3 libros y yo tengo 5?\\
        5! son todas las permutaciones de los elementos que tenía originalmente, y hay que dividir por los que conté de más cuando ubiqué a todos. En este caso calculo las permutaciones y luego les divido lo que conté de más, entonces:
            \begin{equation*}
                \#CP = \underline{5}\cdot\underline{4}\cdot\underline{3} = 60 = \frac{5\cdot4\cdot3\cdot2\cdot1}{2\cdot1} = \frac{5!}{2!}
            \end{equation*}
    \end{enumerate}
\end{defexamples}

\begin{definition}[Variaciones]
    Es la cantidad de subconjuntos ordenados de r elementos que se pueden formar a partir de un conjunto de n elementos. Las variaciones son, de un conjunto total de n elementos son todos los subconjuntos diferentes que puedo extraer si me importa el orden en el que los estoy extrayendo.
    \begin{equation*}
        P_{n,r} = \frac{n!}{(n-r)!}
    \end{equation*}
    \begin{equation*}
        \text{En la calculadora: } \boxed{nPr}
    \end{equation*}
\end{definition}

\begin{defexample}
    En un cine con 70 butacas, ¿de cuántas formas distintas pueden sentarse 45 personas?
    \begin{equation*}
        \#CP = \frac{70!}{(70-45)!} = \frac{70!}{25!}
    \end{equation*}
\end{defexample}

\begin{definition}[Combinaciones]
    Es la cantidad de subconjuntos \underline{NO} ordenados de r elementos que pueden formarse a partir de los conjuntos de n elementos.
    \begin{equation*}
        C_{n,r} = \frac{n!}{(n-r)!\cdot r!} = \binom{n}{r}
    \end{equation*}
    \begin{equation*}
        \text{En la calculadora: } \boxed{nCr}
    \end{equation*}
    En donde $\binom{n}{r}$ se denomina número combinatorio de n elementos tomados de a r. Lo que me dice este número es de cuantas formas puedo sacar r elementos de un total de n cuando no me importa el orden.
\end{definition}

\begin{defexamples} Combinaciones
    \begin{enumerate}
        \item ¿De cuántas formas diferentes puedo elegir 11 personas de un grupo de 30 para formar un equipo de fútbol?\\
        Tengo un conjunto de 30 personas, y voy a elegir un subconjunto de 11 personas entonces n = 30, r = 11 y como el orden no me importa, utilizo el número combinatorio para contar estas personas:
            \begin{equation*}   
                \#CP = \binom{30}{11}
            \end{equation*}
        \item Control de calidad. ¿Cuántas muestras de 10 piezas diferentes puedo elegir de un lote de 100?\\
        Tengo un conjunto, y un subconjunto, y el orden no me interesa, entonces:
            \begin{equation*}
                \#CP = \binom{100}{10}
            \end{equation*}
    \end{enumerate}
\end{defexamples}

\begin{observation}
    \begin{equation*}
        \binom{n}{r} = \binom{n}{n-r}
    \end{equation*}
\end{observation}

\begin{proof}
    \begin{equation*}
        \binom{n}{r} = \binom{n}{n-r}
    \end{equation*}
    \begin{equation*}
        \frac{n!}{r!(n-r)!} = \frac{n!}{(n-r)!r!}
    \end{equation*}
\end{proof}

\begin{defexample} Mazo de 40 cartas esáñolas.
    \begin{enumerate}[label=\alph*)]
        \item En un mazo de cartas españolas, tengo 10 cartas de espada, 10 cartas de copa, 10 cartas de basto y 10 cartas de oro. ¿Cuántas manos diferentes puede tener una persona jugando al truco?\\
            \begin{equation*}
                n = 40, r = 3
            \end{equation*}
            \begin{equation*}
                \#CP = \binom{40}{3}
            \end{equation*}
        \item ¿Cuántas formas distintas hay de recibir una mano de oro?\\
        Todos los casos posibles se dan de elegir 3 cartas de un total de 10 ya que estoy elegiendo entre las cartas que son de oro.
            \begin{equation*}
                \#CP = \binom{10}{3}
            \end{equation*}
        \item ¿Cuántas manos puedo recibir siendo todas las cartas del mismo palo?\\
        Separo en casos: tengo 4 posibles resultados dada la cantidad de palos disponibles, y por cada palo tengo la cantidad de posibilidades del resultado anterior, entonces ahora tengo que sumar todas las formas de sacar 3 de cada palo.
            \begin{equation*}
                \#CP_{oro} = \binom{10}{3}, \#CP_{basto} = \binom{10}{3}, \#CP_{espada} = \binom{10}{3}, \#CP_{copa} = \binom{10}{3}
            \end{equation*}
            \begin{equation*}
                \#CP = \#CP_{oro} + \#CP_{basto} + \#CP_{espada} + \#CP_{copa}
            \end{equation*}
            \begin{equation*}
                \#CP = 4 \cdot \#CP_{palo}
            \end{equation*}
            \begin{equation*}
                \#CP = 4 \cdot \binom{10}{3}
            \end{equation*}
    \end{enumerate}
\end{defexample}

\begin{definition}[Anagramas]
    Dada una palabra, un anagrama es una forma de escribir otra palabra, cambiando las letras de la palabra original de lugar. 
\end{definition}

\begin{defexample}
    Supongamos la palabra "ANANA", ¿cuántos anagramas puedo formar?\\
    ¿Cómo contamos todos los ordenamientos posibles? Usando lo que aprendimos en número combinatorio: tengo 5 posiciones para ubicar 5 letras, las letras que tengo que ubicar son las letras de la palabra "ANANA":
    \begin{enumerate}
        \item Acomodo las letras "A", tengo 3 lugares para ubicarla (porque tengo 3). No me importa el orden.
            \begin{equation}
                \#CP_{A} = \binom{5}{3}
            \end{equation}
        
        \item  Por todas esas posiciones que tengo para ubicar la letra "A", tengo que ver todas las posibles opciones que me quedan para la letra "N". Ahora tengo 2 lugares para ubicarla (porque tengo 2).
            \begin{equation}
                \#CP_{N} = \binom{2}{2} = 1
            \end{equation}
    \end{enumerate}
    Una vez ubicadas las letras "A" tengo sólo una forma posible de ubicar las "N". Si ubico primero las "N", de los 5 lugares que tengo voy a poder usar 2, y de los 3 lugares que quedan voy a elegir 3 para ubicar la "A".
    \begin{equation*}
        \#CP = \binom{5}{3} \cdot \binom{2}{2} = \binom{5}{2} \cdot \binom{3}{3} 
    \end{equation*}
\end{defexample}

\begin{note} Otra forma de pensarlo permutando es hacer el total de permutaciones y dividir esa cantidad por la multiplicación de los casos que conté de más
    \begin{equation*}
        \#CP = \frac{5!}{3!\cdot2!} = \binom{5}{3} = \binom{5}{2}
    \end{equation*}
\end{note}

\begin{defexample} ¿Cuántos anagramas puedo formar de la palabra "MANZANA"?
    \begin{itemize}
        \item Con el método de permutaciones:
            \begin{equation*}
                \#CP = \frac{7!}{1!\cdot3!\cdot2!\cdot1!}
            \end{equation*}
        \item Con números combinatorios (leyendo las letras en orden izq-der):
            \begin{equation*}
                \#CP = \binom{7}{1}\cdot\binom{6}{3}\cdot\binom{3}{2}\cdot\binom{1}{1}
            \end{equation*}
    \end{itemize}
\end{defexample}

\begin{definition}[Permutaciones con elementos repetidos]
    Si tenemos n elementos en los cuales hay $n_{1}$ de la clase 1, $n_{2}$ de la clase 2, ... $n_{k}$ de la k-ésima, el número de permutaciones de $n = n_{1} + n_{2} + \dots + n_{k}$ objetos está dada por
    \begin{equation*}
        \frac{n!}{n_{1}!\cdot n_{2}!\cdot\dots n_{k}!}
    \end{equation*}
\end{definition}

\begin{definition}[Método de bolas y urnas (capitulo1-video3 40:00)]
    Se usa cuando tengo una cantidad de elementos \textbf{indistinguibles}. Si estoy ordenando r elementos indistinguibles en k urnas tengo que dibujar r crucecitas ("$\times$"), y k-1 palitos ("$|$")
\end{definition}

\begin{definition}[Métodos de Bose-Einstein]
    \begin{equation*}
        \#CP = \binom{r + k-1}{r} = \binom{r + k - 1}{k - 1}
    \end{equation*}
\end{definition}

\begin{defexample} Tengo un mapa, donde cada cuadrado es una manzana, y quiero ir del punto C al punto S. ¿De cuántas formas posibles puedo ir de C a S si solo puedo moverme hacia la izquierda o abajo.\\
Esto es análogo al problema del anagrama, por lo tanto
    \begin{equation*}
        \#CP = \binom{10}{3} = \binom{10}{7}
    \end{equation*}
\end{defexample}

\end{document}