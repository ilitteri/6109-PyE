\documentclass[../main.tex]{subfiles}

\subsection{Esperanza de una Variable Aleatoria}
\subfile{subsections/esperanza_variable_aleatoria.tex}

\begin{property}
    Sea $X$ una $V.A.$ con $E(X) = \mu$ si $h(X) = aX + b \rightarrow E(h(X)) = a\mu + b$.
\end{property}

\begin{proof}[Para V.A.C.]
    \begin{equation*}
        \begin{aligned}
            E(h(X)) &= E(aX + b)\\
                    &= \sum_{x \in R_{X}} (ax + b) p_{X}(x)\\
                    &=  \sum_{x \in R_{X}} (axp_{X} + bp_{X})\\
                    &= a \underbrace{\sum_{X \in R_{X}} x \cdot p_{X}(x)}_{\mu} + b \underbrace{\sum_{X \in R_{X}}\cdot p_{X}(x)}_{1}\\
                    &= a\mu + b
        \end{aligned}
    \end{equation*}
\end{proof}

\subsubsection{Caso General}
Sea $X$ una $V.A.$ con función de distribución $F_{X}(x) = P(X \leq x)$, si $h(X)$ es una función cualquiera de $X$, si definimos $A$ como el conjunto de átomos (valores de $X$ que concentran masa positiva), entonces
\begin{equation*}
    E(h(X)) = \sum_{x \in A}h(x) \cdot P(X = x) + \int_{\mathbb{R}\backslash A} h(x) \cdot F'_{X}(x) dx
\end{equation*}

\begin{itemize}
    \item Si $X$ es una $V.A.D.$, en este caso el conjunto de átomos acumula probailidad 1, y la derivada de $F_{X}(x)$ va a ser 0 en todo el resto de los reales, entonces el segundo término desaparece, y como la suma de las probabilidades de todos los elementos de A es 1, $P(X = x) = p_{X}(x)$.
    \item Si $X$ es una $V.A.C.$, el conjunto de átomos está vacío, osea que el primer término desaparece y la deribada de la función de distribución existe y la llamamos función de densidad
    \begin{equation*}
        \frac{\partial{F_{X}(x)}}{\partial{x}} = f_{X}(x)
    \end{equation*}
\end{itemize}

\begin{example*}
    Sea $X$ una $V.A.$ con
    \begin{equation*}
        F_{X}(x) = \frac{x}{4} \cdot \mathbf{1}\{0 \leq x < 1\} + \frac{1}{3} \cdot \mathbf{1}\{1 \leq x < 2\} + \frac{x}{7} \cdot \mathbf{1}\{4 \leq x\}
    \end{equation*}
    \begin{enumerate}
        \item Hallar $E(X)$.
        \item Hallar $E(X|X<1)$ y $E(X|X\leq1)$
    \end{enumerate}
    Solución: lo primero que hacemos es graficar la función de distribución.
    Al ver que se trata de una variable aleatoria mixta, lo que sigue hacer es identificar el conjunto de átomos (que está formado por todos los puntos de discontinuidad de la función, es decir, donde la función pega saltos) en este caso
    \begin{equation*}
        A_{X} = \{1, 4\}
    \end{equation*}
    y luego anotamos sus probabilidades fijándonos lo que mide un salto
    \begin{equation*}
        \begin{aligned}
            P(X = 1) = \frac{1}{12}\\
            P(X = 4) = \frac{1}{3}\\
        \end{aligned}
    \end{equation*}
    \begin{enumerate}
        \item Vamos a calcular $E(X)$ usando la última fórmula
        \begin{equation*}
            \begin{aligned}
                E(X) &= \sum_{x \in A_{X}} + \int_{-\infty}^{\infty} x \cdot F'_{X}(x) dx\\
                     &= 1 \cdot P(X = 1) + 4 \cdot P(X = 4) + \int_{0}^{1} x \frac{1}{4} dx + \int_{2}^{4} x \frac{1}{6} dx\\
                     &= \frac{1}{12} + \frac{4}{3} + \frac{1}{8} + 1\\
                     &= \frac{61}{24}\\
                     &\approx 2.54
            \end{aligned}
        \end{equation*}
        Un detalle muy importante cuando calculamos la esperanza, es que debe estar dentro del soporte de la variable.
        \item La esperanza se calcula a una $V.A.$ En este caso, una $V.A.$ truncada. Para este caso, al truncar, transformo mi variable, que antes era mixta, a una continua, luego existe su función de densidad
        \begin{equation*}
            f_{X|X<1}(x) = \frac{1}{P(X<1)} \frac{\partial{F_{X}(x)}}{\partial{x}} \cdot \mathbf{1}\{X<1\}
        \end{equation*}
        Entonces la esperanza que nos piden la calculo como la esperanza de una variable aleatoria continua
        \begin{equation*}
            \begin{aligned}
                E(X|X<1) &= \int_{-\infty}^{\infty} x \cdot f_{X|X<1}(x) dx\\
                         &= \frac{1}{P(X<1)} \underbrace{\int_{0}^{1} x \frac{1}{4} dx}_{E(X\mathbf{1}\{X<1\}), \text{Esperanza Parcial}}\\
                         &= \frac{\frac{1}{8}}{\frac{1}{4}}\\
                         &= \frac{1}{2}
            \end{aligned}
        \end{equation*}
        Luego con la fórmula de esperanza parcial calculo la ultima parte
        \begin{equation*}
            \begin{aligned}
                E(X|X\leq1) &= \int_{-\infty}^{\infty} x \cdot f_{X|X\leq1}(x) dx\\
                         &= \frac{\frac{1}{12}\int_{0}^{1} x \frac{1}{4} dx}{F_{X}(1)} \\
                         &= \frac{\frac{1}{12}+\frac{1}{8}}{\frac{1}{3}}\\
                         &= \frac{5}{8}
            \end{aligned}
        \end{equation*}
    \end{enumerate}
\end{example*}

\subsection{Varianza de Una Variable Aleatoria}
\subfile{subsections/varianza_variable_aleatoria.tex}

\subsection{Desvío Estándar}
\subfile{subsections/desvio_estandar.tex}

\subsection{Mediana}
\subfile{subsections/mediana.tex}

\subsection{Moda}
\subfile{subsections/moda.tex}

\subsection{Esperanza de un Vector Aleatorio}
\subfile{subsections/esperanza_vectores_aleatorios.tex}

\subsection{Covarianza}
\subfile{subsections/covarianza.tex}

